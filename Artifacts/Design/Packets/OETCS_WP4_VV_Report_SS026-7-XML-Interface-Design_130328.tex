\documentclass[a4paper]{article}
%\documentclass[a4paper,german]{article}
\usepackage{graphicx}
\usepackage{xspace}
\usepackage{longtable}
\usepackage{array}
\usepackage{xifthen}% provides \isempty test

\usepackage{color}   %May be necessary if you want to color links
\usepackage{hyperref}
\hypersetup{
    colorlinks=true, %set true if you want colored links
    linktoc=all,     %set to all if you want both sections and subsections linked
    linkcolor=blue,  %choose some color if you want links to stand out
}


\pagestyle{myheadings}
\markright{openETCS \hfill {\tiny This work is licensed under a Creative Commons Attribution-ShareAlike 3.0 Unported License} \hfill}


%\usepackage{amsmath}
%\usepackage{amssymb}
% % Deutsche Silbentrennung
% \usepackage[ngerman]{babel}
% % Deutsche Umlaute
% \usepackage[ansinew]{inputenc}
% %\usepackage[latin1]{inputenc}


\setlength{\parindent}{0pt}
\setlength{\parskip}{3pt}

% editing

% Starts a new line nearly everywhere
\newcommand{\nl}{\mbox{}\\}

%Texts in a box (eg. for comments)
% Short text (no line break) 
\newcommand{\cmmnt}[1]{\framebox{#1}}
% Long text (separate lines
\newcommand{\bgcmmnt}[1]{\nl\framebox{\parbox{.95\textwidth}{#1}}\nl[2mm]}

%Uncomment for getting rid of comments in output
%\renewcommand{cmmnt}[1]{}
%\renewcommand{\bgcmmnt}[1]{}

% Macros for minutes
\newcommand{\Q}[2]{\paragraph{Question} 
	\ifthenelse{\isempty{#1}}%
    	{}% if #1 is empty
    	{by #1}% if #1 is not empty
    : #2}
\newcommand{\A}[2]{\newline{\textbf{Answer}}
	\ifthenelse{\isempty{#1}}%
    	{}% if #1 is empty
    	{by #1}% if #1 is not empty    
    : #2}
\newcommand{\C}[2]{\textbf{Comment} by #1: #2}

% End of document marker
\newcommand{\eod}{\rule{\textwidth}{1pt}\nl \textit{End of Document}}

\begin{document}
\title{openETCS Verificationreport \\ on Subset-026-7 XML design file\\}
\author{Marc Behrens}
\date{Version 01, 2013-03-28}

\setlength{\oddsidemargin}{0mm}
\setlength{\textwidth}{150mm}

%\pagestyle{empty}

\maketitle

\newline \mbox{}
\newline \mbox{}
\newline \mbox{}
\newline \mbox{}
\newline \mbox{}
\newline \mbox{}
\newline \mbox{}
\newline \mbox{}
\newline \mbox{}
\newline \mbox{}

\section*{Document Control}

\begin{tabular}{|l|r|*{2}{p{.3\textwidth}|}}
\hline
\multicolumn{4}{|l|}{\texttt{'OETCS\_WP4\_VV\_Report\_SS026\-7\-XML\-Interface\-Design\_130328.tex'}}
\\\hline
\textbf{Version} & \textbf{Date} & \textbf{Author of report} & \textbf{Changes/Comment}
\\\hline
01 & 2013-03-28 & Marc Behrens & All sections  
\\\hline
\end{tabular}

Document validated by?

\pagebreak

%\renewcommand{\contentsname}{Content}
\label{sec:agenda}
\tableofcontents

\pagebreak


\section{Artifact Verified}

\begin{tabular}{|l|r|*{2}{p{.3\textwidth}|}}
\hline
\multicolumn{4}{|l|}{\texttt{'pacANDvar_Attributes.xml'}}
\\\hline
\textbf{Version} & \textbf{Date received} & \textbf{Author of file} & \textbf{git link to file}
\\\hline
01 & 2013-02-25 & G.Assmann & ??git-link??
\\\hline
\end{tabular}

\section{Further Applicable Documents}

\begin{tabular}{|l|r|*{2}{p{.3\textwidth}|}}
\hline
\multicolumn{4}{|l|}{\texttt{'Subset\_Att\_26\_7.xsd'}}
\\\hline
\textbf{Version} & \textbf{Date received} & \textbf{Author of file} & \textbf{git link to file}
\\\hline
01 & 2013-02-25 & G.Assmann & ??git-link??
\\\hline
\end{tabular}


\section{Specific Artifact Information}

\medskip\noindent%

\begin{tabular}{|l|r|r|r|}

  \hline
\textbf{format} & \textbf{generated by tool} & \textbf{used by tool} & \textbf{used for}
\\\hline
XML & XML-Spy & XML-Spy & Bitwalker\\\hline
\end{tabular}



\section{Relevant Specifications}

\medskip\noindent%

\begin{tabular}{|l|r|r|}

  \hline
\textbf{specificaiton title} & \textbf{version} & \textbf{release date}
\\\hline
Subset-026-7 & 3.3.0 & 2012-03-07
\\\hline
\end{tabular}



\section{Type of Verification}

\medskip\noindent%

\begin{tabular}{|l|r|r|r|}

  \hline
\textbf{Type} & \textbf{tool used} & \textbf{version tool used} & \textbf{release date tool} 
\\\hline
manual review & gVIM & 7.3 & 2010-15-08
\\\hline
\end{tabular}





\pagebreak

% Use several tabular environments to split long result lists over pages

%\setcounter{section}{0}
%\setcounter{subsection}{0}
%use sections to document the agenda
\section{Verification Results} %1
\subsection{Executive Summary} %1.1
\setlength{\extrarowheight}{1.5pt}
Within the openETCS project WP4
Current 
Reference http://www.openetcs.org

\subsection{How the verification is executed} %1.2
The type of verification used is manual review.

\subsection{Findings} %1.2


\begin{longtable}{|p{0.83\textwidth}|p{.02\textwidth}|p{.15\textwidth}|}
% Description (free text)
% A (action item) OR D (decision) OR F (fact/finding) 
% responsible (for action items)
% 	deadline (for action items)
%
% in case of multi-page tables use the following sequence to end one page:
%		& T & Author
%		\end{longtable}
%        		\vskip 1 cm 
%		        \clearpage
%	    \begin{longtable}{|p{0.83\textwidth}|p{.02\textwidth}|p{.15\textwidth}|}
%	    \vskip 1 cm 
% header ------------------------
\hline
\textbf{Description} & \textbf{T} & \textbf{Resp.} 
%\hline
\endhead
\hline

& 
& 
\\\hline

\end{longtable}
%\vskip 1 cm 
%\clearpage


\begin{longtable}{|p{0.83\textwidth}|p{.02\textwidth}|p{.15\textwidth}|}
\hline
\textbf{Description} & \textbf{T} & \textbf{Resp.} 
%\hline
\endhead
\hline

& 
& 
\\\hline
\end{longtable}

\bgcmmnt{Glossary:
\begin{description}
	\item[XX] 
	\item[XY]
	\item[XZ]
\end{description}}



\section*{Notes}

This format lacks references to ITEA~2 so far.

% Optional for additional free text
  
\eod


 

\end{document}