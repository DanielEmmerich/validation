\documentclass{article}

\usepackage{verbatim}


\title{Verification Report for Architecture and Design of Train Positioning \\Version 0.1}
\author{Marc Behrens (DLR), Bernd Gonska (DLR),\\ Jens Gerlach (Fraunenhofer), Bernd Hekele (DB),\\ Jan Welte (TU-BS)} 
\date{Oct 29, 2014}

%based on VnVRprtTmplt-131101-02.tex by Hardi Hunger

\newcommand{\tbi}[1]{$<$\textit{#1}$>$}

% Starts a new line nearly everywhere
\newcommand{\nl}{\mbox{}\\}
\newcommand{\nlskip}[1]{\mbox{}\\[#1]}

%
%Comments
\newcommand{\cmmnt}[1]{\framebox{#1}}
\newcommand{\bgcmmnt}[1]{\nl\framebox{\parbox{.95\textwidth}{#1}}\nl[2mm]}
%\renewcommand{\bgcmmnt}[1]{}
%

\newcommand{\eod}{\nl\rule{.95\textwidth}{1pt}\nl\textit{End of Document}}

\begin{document}
\maketitle

\begin{abstract}

This verification report presents the verification results for the architecture, interfaces and design artifacts for the component "Train Positioning" in the overall openETCS Kernel architecture.

\begin{comment}
This template provides the required content to complete the verification of architecture and design artifacts.
To close the development phase for this artifact all required information shall be given, even if it can only be stated that specific aspects are missing in the artifact due to open points in related artifacts. 

The template should be used as a guideline to check whether all
information is given appropriately. The wording used in this proposal
is by no means mandatory. And if you feel that more information is
useful to describe your activity within the context of openETCS, you
should of course do so. Feel free to add additional categories of
description as adequate. 

Also the \LaTeX{} macros may be changed, though the use of
\texttt{paragraph} and \texttt{subparagraph} enables easy integration
into higher-level documents (they are not numbered automatically,
which may be a drawback in other respects). 
\end{comment}
\end{abstract}

\section{Verification of \tbi{architecture and design artifact (component)}}

\paragraph{Roles}

\begin{itemize}
\item Fausto - Design (PM)
\item Jens - Verification of SW/Implementation
\item Design/Implementation
\item Jan Welvaarts - Design
\item Vincent - Simulation/Design
\item Marc - Verification
\item Jan Welte - Verification
\item Bernd Hekele - Verification
\end{itemize}

\subsection{Identity and Configuration of Verification Object}

This sections shall provide all basic information defining the \tbi{verification object}.

\tbi{architecture and design artifact(component)}

\tbi{designers}

\tbi{Covered Software Requirement Specifications}

\tbi{Related Systems/Components}


\subsection{Software Architecture and Interface Verification}

This sections shall provide all verification results concerning the architecture and interface for the verification object. 

\bgcmmnt{For all verification aspects addressed in the following section the following 3 points shall be state clearly:
\begin{enumerate}
\item Responsible verifier
\item Use verification strategy and technique (with reference to the V\&V Plan)
\item Verification results (level of conformity, detected errors or deficiencies and made assumptions)
\end{enumerate}}

The following aspects have to be verified (in accordance with EN 50128 7.3.4.42):

\subsubsection{Internal Consistency}

Are the internal functional allocation and all related input and output consistent?

\subsubsection{Adequacy to fulfill Software Requirements}

Are the listed functions and all input and outputs adequate to cover the intended Software Requirements?
Therefore the following 2 aspects shall be assess:

\begin{itemize}
\item Consistency
\item Completeness
\end{itemize}

\subsubsection{Readability and Traceability}

Are all related system and software requirements uniquely referenced and is the relationship to other documents clearly defined?
Are all parts of the architecture and inputs and outputs referenced to the related requirements.
Are the elements referred to in the same way in all documents?

\subsubsection{Consideration of hardware and software constraints}

Does the architecture and interface respect constrains of the software development technique?
Does the architecture and interface respect existing hardware constraints?

\subsection{Software Design Verification}

This sections shall provide all verification results concerning the design specification for the verification object. 

\bgcmmnt{For all verification aspects addressed in the following section the following 3 points shall be state clearly:
\begin{enumerate}
\item Responsible verifier
\item Use verification strategy and technique (with reference to the V\&V Plan)
\item Verification results (level of conformity, detected errors or deficiencies and made assumptions)
\end{enumerate}}

The following aspects have to be verified (in accordance with EN 50128 7.3.4.42):

\subsubsection{Internal Consistency}

Is the design consistent in itself?

\subsubsection{Adequacy to fulfill Software Requirements}

Is the design and its variables adequate to cover the architecture and all stated interfaces? Does the design cover all intended Software Requirements?
Therefore the following 2 aspects shall be assess:

\begin{itemize}
\item Consistency
\item Completeness
\end{itemize}

\subsubsection{Readability and Traceability}

Are all related system and software requirements uniquely referenced and is the relationship to other documents clearly defined? 
Are all design element referenced to the related requirements? Are all references to the architecture and their interfaces given?
Are the elements referred to in the same way in all documents?

\subsubsection{Consideration of hardware and software constraints}

Does the architecture and interface respect constrains of the software development technique?
Does the architecture and interface respect existing hardware constraints?






\end{document}