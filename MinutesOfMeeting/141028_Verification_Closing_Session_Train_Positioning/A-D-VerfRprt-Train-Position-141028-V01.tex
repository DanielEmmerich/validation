\documentclass{article}

\usepackage{verbatim}
\usepackage{hyperref}
\usepackage{fixme}

\title{Verification Report for Architecture and Design of Train Positioning \\Version 0.1}
\author{Marc Behrens (DLR), Bernd Gonska (DLR),\\ Jens Gerlach (Fraunenhofer), Bernd Hekele (DB),\\ Jan Welte (TU-BS)} 
\date{Oct 29, 2014}

%based on VnVRprtTmplt-131101-02.tex by Hardi Hunger

\newcommand{\tbi}[1]{$<$\textit{#1}$>$}

% Starts a new line nearly everywhere
\newcommand{\nl}{\mbox{}\\}
\newcommand{\nlskip}[1]{\mbox{}\\[#1]}

%
%Comments
\newcommand{\cmmnt}[1]{\framebox{#1}}
\newcommand{\bgcmmnt}[1]{\nl\framebox{\parbox{.95\textwidth}{#1}}\nl[2mm]}
%\renewcommand{\bgcmmnt}[1]{}
%

\newcommand{\eod}{\nl\rule{.95\textwidth}{1pt}\nl\textit{End of Document}}

\begin{document}
\maketitle

\begin{abstract}

This verification report presents the verification results for the architecture, interfaces and design artifacts for the component "Train Positioning" in the overall openETCS Kernel architecture.

\begin{comment}
This template provides the required content to complete the verification of architecture and design artifacts.
To close the development phase for this artifact all required information shall be given, even if it can only be stated that specific aspects are missing in the artifact due to open points in related artifacts. 

The template should be used as a guideline to check whether all
information is given appropriately. The wording used in this proposal
is by no means mandatory. And if you feel that more information is
useful to describe your activity within the context of openETCS, you
should of course do so. Feel free to add additional categories of
description as adequate. 

Also the \LaTeX{} macros may be changed, though the use of
\texttt{paragraph} and \texttt{subparagraph} enables easy integration
into higher-level documents (they are not numbered automatically,
which may be a drawback in other respects). 
\end{comment}
\end{abstract}

\section{Roles}

\begin{itemize}
\item Fausto - Design (PM)
\item Jens - Verification of SW/Implementation
\item Design/Implementation
\item Jan Welvaarts - Design
\item Vincent - Simulation/Design
\item Marc - Verification
\item Jan Welte - Verification
\item Bernd Hekele - Verification
\end{itemize}


\section{Verification Object}

\input{verification_object}


\section{Software Architecture, Interface and Design Verification}

This section presents all verification results concerning the verification object \texttt{Train Positioning}. 

\begin{comment}
\bgcmmnt{For all verification aspects addressed in the following section the following 3 points shall be state clearly:
\begin{enumerate}
\item Responsible verifier
\item Use verification strategy and technique (with reference to the V\&V Plan)
\item Verification results (level of conformity, detected errors or deficiencies and made assumptions)
\end{enumerate}}
\end{comment}

The following subsection present all different verification aspects in accordance with EN 50128 7.3.4.42.

\subsection{Internal Consistency}

\textit{by Jan Welte and Marc Behrens}



Contant:
\begin{itemize}
\item relations
\item historical development
\item claim of same approach
\item of naming between documents consistence 
\end{itemize}

Are the internal functional allocation and all related input and output consistent?


\subsection{Adequacy to fulfill Software Requirements}
by Bernd Gonska

\subsubsection{Description of the developed train positioning system}
The train positioning system clearly deviates from the SRS on purpose. It is intended increase safety and accuracy.

It basically implements the following concepts:
\begin{itemize}
\item All distances are given as the triple of safe distances (minimum,nominal,maximal)

\item The estimated position, (also named nominal position) is calculated to be the middle of the maximum safe position and the minimum safe position.

\item Each BG has a an own accuracy and position, relative to the LRBG and the LRBG accuracy. Locations do not change their reference BG.
  
\item Linking distances and accuracies are used to improve the accuracy when ever possible.

\item Accuracy of a distance is calculated in a restrictive safe manner. For example: The accuracy of the distance between two ends of a linking chain includes the first and the last BG accuracy. The distance between two BG without linking is the odometry measured distance. The accuracy is the odometry acuracy during that distance. This inaccuracy is not reseted later.

\item The confidence interval of an announced location never increases when a new BG is accepted. Always use the most accurate information. The odometry error estimation is trustworthy enough to optimize linking accuracy and distances.
\end{itemize}

\subsubsections{Deviations}
3.6.4.1 REMARK: There are several confidence intervals: They depend on the announced location.//

3.6.4.2: In addition, the odometry inaccuracy of older track areas and older linking accuracy can be taken into account to widen a the confidence interval for safety reasons. The location accuracy of the LRBG is shortened on if the detected Balise group position is extreme. An old confidence interval can be taken instead of a larger new one.//

3.6.4.2.2: An odometer inaccuracy may not be reset at the new LRBG.//

3.6.4.3: Even if the linking chain is not complete, linked parts replace odometry distances if they provide higher accuracy.

In the "sandwich problem" where two linked BGs enclose an unlinked BG 
(linked BG1 -> unlinked BG2 -> linked BG3) 
the distance and the accuracy between BG2 and BG3 can be calculated involving the linking accuracy of BG1and BG3 and the linking distance between BG1 and BG3. 

The estimated distance may differ from the linking distance and from the odometry measured distance. The estimated distance is set to the middle of the maximum and minimum safe distance.

3.6.4.3.1:
The train takes responsibility, it does not reset inaccuracies if this could lead to unsafe behavior. 

Figure 13 a,b,c:
There is more than one confidence interval.

The confidence interval is calculated differently.

The estimated distance can be different since it is the middle of the maximum and minimum safe distance.

Linking distance is not used if it leads to a less accurate distances.

3.6.4.4:
The estimated distance is the middle of the maximum and the minimum safe position with respect to the possible LRBG position. It may differ from the measured traveled distance.

3.6.4.4.1:
analogue for the rear end position.

3.6.4.7.1:
The unlinked BG confidence interval is not reset at the next LRBG.

3.6.4.7.2
The unlinked BG confidence interval is not reset at the next unlinked BG. In some cases the estimated traveled distance between two unlinked BG is calculated by using other rules.


\subsubsection{Readability and Traceability}

by Marc Behrens

Content
\begin{itemize}
\item traceability of requirements
\item unique references
\end{itemize}


Are all related system and software requirements uniquely referenced and is the relationship to other documents clearly defined?
Are all parts of the architecture and inputs and outputs referenced to the related requirements.
Are the elements referred to in the same way in all documents?

** 10' item: look for findings inside the two verification reports
**  10' structure: two chapters, one for each report
** 20' 3 documents: readability: statistics: wordlength + syllable over sentence length
** 20' look in 2.x: traceability to openetcs req
** 10' only 3.6 available traceability to SRS requirements
*** high level
** traceability to TSI requirements
*** 20' are there more req in TSI to positioning?
*** 10' what is missing
*** 15' design reasoning: what is needed for performance resons
*** 5' what is the least cycle time
*** 5' are realtime requirements defined on architecture level?
** 10' traceability to higher level artifacts
*** 10' high level requirements were defined during workshop as RO US 
  who will be response 
** 10' application of glossary
*** 5' subset-023  
*** 5' openETCS glossary
*** 5' make jan's document openETCS licensed + upload



\subsubsection{Consideration of hardware and software constraints}



Hardware design is out of scope of the openETCS project. 
The considerations related to hardware are planned to be based on assumptions which have not yet been formulated.

\noindent 
Software constraints encompass 

\begin{itemize}
   \item restrictions implied by the coding standards and software design method
   \item timing\slash performance constraints
   \item memory constratints 
   \item constraints implied by 
         the interfacing system (e.g. decoder and encoder functions)
   \item the operating system
\end{itemize}




\bibliographystyle{plain}
\bibliography{verification.bib}

\end{document}