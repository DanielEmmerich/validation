\documentclass[a4paper,german]{article}
%%BeginIpePreamble

\usepackage{amsmath}
\usepackage{amssymb}
% % Deutsche Silbentrennung
% \usepackage[ngerman]{babel}
% % Deutsche Umlaute
% \usepackage[ansinew]{inputenc}
% %\usepackage[latin1]{inputenc}
\usepackage{graphicx}
\usepackage{xspace}
\usepackage{eurosym}
%Binds \euro to some version of the Euro Symbol

 
\setlength{\parindent}{0pt}
\setlength{\parskip}{3pt}

% editing
\newcommand{\itbox}[1]{\mbox{\it #1\/}}
%\newcommand{\itbox}[1]{\mathit{#1}}
\newcommand{\abbrev}[1]{{\small #1}}
% Starts a new line nearly everywhere
\newcommand{\nl}{\mbox{}\\}
\newcommand{\nlskip}[1]{\mbox{}\\[#1]}
\newcommand{\eod}{\rule{\textwidth}{1pt}\nl \textit{End of Document}}

%Uncertain items
\newcommand{\qq}[1]{?`#1?}

%
%Comments
\newcommand{\cncpt}[1]{\framebox{#1}}
\newcommand{\bgcncpt}[1]{\nl\framebox{\parbox{.95\textwidth}{#1}}\nl[2mm]}
%\renewcommand{\bigconcept}[1]{}
%
%Around formulas
\newcommand{\punctsep}{\hspace{.5ex}}

%%%%%%%%%%%%%%%%%%%%%%%%%%%%%%%%%%%%%%%%%%%%%%%%%%%%%%%%%%%%%%%%%%%%%%%%%%

% Date of writing
\newcommand{\creationDateA}{140218}
% \newcommand{\creationDateB}{JJMMTT}
% \newcommand{\creationDateC}{JJMMTT}

% event name (part of file name)
\newcommand{\eventDescription}{IntrnlAssssmnt}

% Version number (part of file name)
\newcommand{\versionNumber}{01}

% Start date of meeting
\newcommand{\meetingStartDate}{\creationDateA}
\newcommand{\meetingStartTime}{13:00}
% End date of meeting
\newcommand{\meetingEndDate}{140219}
\newcommand{\meetingEndTime}{15:00}

%%%%%%%%%%%%%%%%%%%%%%%%%%%%%%%%%%%%%%%%%%%%%%%%%%%%%%%%%%%%%%%%%%%%%%%%%%

\begin{document}
\title{Protocol openETCS Internal Assessment Workshop}
\author{Hardi Hungar}
\date{Version \versionNumber, 20\creationDateA}

%\pagestyle{empty}

\maketitle
%
\begin{abstract} 

\end{abstract}

\section*{Document Control}

\begin{tabular}{|l|r|*{2}{p{.3\textwidth}|}}
\hline
\multicolumn{4}{|l|}{\texttt{PR\_TS\_\eventDescription\_\creationDateA\_\versionNumber.tex}}
\\\hline
\textbf{Version} & \textbf{Date} & \textbf{Changes} & \textbf{Comment}
\\\hline
01 & \creationDateA & All sections & Initial 
\\\hline
\end{tabular}

\section*{Meeting Data}
\label{sec:rahmendaten}

\begin{tabular}{|l|r|r|}
\hline
 \textbf{Start} & \meetingStartDate & \meetingStartTime
\\\hline
 \textbf{End} & \meetingEndDate & \meetingEndTime
\\\hline
\end{tabular}

\medskip\noindent%

\begin{tabular}{|l|r|r|l|}
  \hline
\textbf{Participant} & \textbf{Initials} & \textbf{Institution} & \textbf{Position}
\\\hline
% Name & Initials & Institution %& Position/Role
%\\\hline
Jan Welte & JW &TU-BS &
\\\hline
Hansj�rg Manz & HM & TU-BS &
\\\hline
Hardi Hungar & HH & DLR &
\\\hline
Izaskun de la Torre & IT & SQS & 
\\\hline
Marc Behrens & MB & DLR &
\\\hline
Norbert Sch�fer& NS & AEbt &
\\\hline
Stefan Jagusch & SJ & AEbt &
\\\hline
Jens Gerlach & JG & Fraunhofer &
\\\hline
Ralf Pinger & RP & Siemens &
\\\hline
Luca Macchi & LM & RINA &
\\\hline
Fr\'ed\'erique Vall\'ee & FV & All4TEC &
\\\hline
Anthony Faucogney & AF &All4TEC &
\\\hline
Bernd Hekele & BH & DB &
\\\hline 
Klaus-R�diger Hase & KRH & DB &
\\\hline
\end{tabular}

\section*{Day 1 (18.02.2014)}
\label{sec:day-1-18.02.2014}

\section{Safety and Process}
\label{sec:safety-process}

\subsection{Safety Activities (JW)}
\label{sec:safety-activ-jan}

\begin{itemize}
\item Safety is a system property---the SW itself cannot be declared
  ``safe'' unless put into relation with the HW.
\item The hazard list plays an important role in all safety activities 
\item The safety activities are tied to the artefacts produced---the
  parts of the design, their verification and validation.
\item The safety case shall be written in Goal Structure Notation
  (GSN), linking the evidence.
\item 
\end{itemize}

\subsection{Current State SSRS (System Analysis---JW)}
\label{sec:current-state-ssrs}

\begin{itemize}
\item Table of System Functions
\item Evaluation of function description in Subset~026---is it
  sufficient to start modeling (usually not)
\item Two functions selected: Within train location functionality
  \begin{itemize}
  \item read balise message
  \item \qq{which}
  \end{itemize}
\item Detail by sketching a SysML (Papyrus) model
\end{itemize}


\subsection{Safety Analysis Approach}
\label{sec:safety-analysis}

\begin{itemize}
\item Demonstration on Siemens design part (Management of Radio Communication)
\item Hazard Identification performed
\item FMEA performed, 18 safety criteria defined (text table),
  established tracing to Subsert~026
\item \textbf{Preliminary conlcusion}:  This approach will take up
  more ressources than available if pursued.
\end{itemize}

\subsection{The openETCS Process (Bernd Hekele)}
\label{sec:open-proc-bernd}

\begin{itemize}
\item Important Links
  \begin{itemize}
  \item Development Process (aligned with Eclipse
  \item Quality plan
  \item \ldots
  \item \ldots
  \end{itemize}
\item Make use and unify several different skills and approaches in openETCS
\item Whole process: A standard V development process, but:
  \begin{itemize}
  \item It defines an ideal final outcome of the development activities
    (which cannot be achieved with the project resources)
  \item it does not (and cannot, as it defines the final outcome) the
    way to get there. This can be roughly seen from the WP~3 backlog.
  \item First, the process will be instantiated for some small part of
    the funcitonality, from SSRS down to the SCADE model and including V\&V
  \item Then, this willbe iterated for a larger functionality chunk
  \item There will be by mid 2014 a detailed version of the modeling
    work plan (building on Alstom's legacy)
  \end{itemize}
\item Agile/SCRUM work organisation---this is not in line with the
  ``usual'' CENELEC development approaches. But it is faster in
  progress and permits retargeting during development (which by chance
  fits well the needs of the project which has not got the time to
  first stabilize requirements)
\item \textbf{FV:} We have a problem in the project, because
  Subset~026 does not suffice as a specification and more or less
  everybody in the project is waiting on the SSRS. At the moment, the
  internal assessment cannot progress any further.
\item \textbf{NS:} There are too many processes in openETCS, and they
  are not consistent. \textbf{BH:} Quality is maybe more of a concern,
  with a highly diverse project team. \textbf{SJ:} AeBt did not
  encounter any agile development in practice. \textbf{LM:} In the
  end, one has to demonstrate that the requirements are
  met. \textbf{AF:} There was already some successful agile
  development (DO178). 

\end{itemize}

\section{Documents Assessed}
\label{sec:documents-assessed}

\subsection{QA Plan (IdlT)}
\label{sec:qa-plan}

\begin{itemize}
\item In the current (2014-02-18) version of the QA plan, five issues raised
in the internal assessment have been fixed. The other issues cannot be
fixed at the current state of the project.
\item The life cycle description needs some refinement
  lateron.
\item Roles and independence too.
\item As methods and tools are not yet selected, the choice can
  currently not be justified (what would need to be done in the QA plan)
\item The same applies to many other issues
\end{itemize}


\textbf{NS:} There is just one standard to observe:
CENELEC~50128:2011. This standard is rather general, According to the
CENELEC, several plans need to be written. Each one should name the
documents to produce, later the authors. These plans should be
specific, clear in stating what has to be done and by whom, and very
detailed.  

\section{Internal Assessment (FV)}
\label{sec:intern-assessm-fred}

\begin{itemize}
\item The main problem was that there was no project to assess
\item Five documents were assessed. For each, the quality was assessed
  and recommendations wrt.\ achieve CENELEC conformance
\item Doceument control process ok
\item Revision and review process considered out of scope
\item QA plan considered insufficient.
  \begin{itemize}
  \item It does not include the
  lifecycle (the V lifecycle which is to be considered as the end
  result, just produced in an agile way).
  \item The safety plan should be included/referenced by the QA plan,
  but there was none at that time.
  \item The tools, even if no final decision has been taken, can be described.
  \item The models built and verification attempts are interesting,
    but it is not clear how they will ever grow to something coherent
  \end{itemize}
\item The SCMP and the CPMP have been written by different
  organisations without the necessary coordination
  \begin{itemize}
  \item The SCMP is generally acceptable, but needs to be reconciled
    with the CENELEC in some places, and it needs to be instantiated
    to the project. 
  \end{itemize}
\item \textbf{JW:} How to proceed with the safety plan? \textbf{FV:}
  The documents should be defined in the QA plan. How to produce them
  is part of the safety plan.
\item In the QA plan, the methods and techniques chosen for design and V\&V
  activities are to be listed. The justification for the selection has
  to be elsewhere (eg, the safet. Currently, one could try to name the
  ones responsible for the selection. 
\end{itemize}

\subsection{Internal Assessment Summary (SJ, MB)}
\label{sec:intern-assessm-summ}

\begin{itemize}
\item \textbf{SJ:} The planning documents need input from the WPs, and
  the WPs need guidance from someone to plan the project.
\item \textbf{HH:} Usually, there should be a small group of informed
  people taking the main decision and detail the plan to a degree that
  different approaches roughly fit in (bottom-up).
\item \textbf{FV:} It is not possible to assess many different approaches.
\item There is the Papyrus/SCADE toolchain, and there are other
  approaches (B, ETFMS, ERTMS spec model). How ``official'' is the
  Papyrus/SCADE one?
\item The overall project management needs to find a solution to
  install a coherent approach. Eg.:  A lot of participants working in
  the official approach, with a few exploring the alternatives. This
  would emphasize the development aspect of openETCS as opposed to the
  research aspect (without completely abandoning the latter).
\item \textbf{JW:} There are named responsibilities in the project for
  some tasks, and these should be acted upon. Some are missing
  (according to JW, \qq{which---not named}).
\item There is a problem with managing openETCS, as it is difficult to
  get all involved contributors in line.
\item \textbf{FV:} The QA plan should include the life cycle as
  presented by JW. Methods and techniques should go in there, the
  perimeter of the project (what will be considered as demonstrating
  the openETCS devlopment process, methods and tools).
\item The goal for the assessment should be to classify in which
  respect the project approach achieved compliance.
\item \textbf{RP:} As an example: The Bitwalker could be formally verified, and the
  assessor should say eg.\ that it would be ok if the tools were T2
  qualified.
\item \textbf{SJ:} Merlin has already put checklists on github
  detailing what needs to be checked during an assessment. Roughly,
  first the process documentation is looked at, then an audit of open questions . This would
  be very expensive to perform for openETCS. One could simplify this
  to an assessment of (explained) documentation.
\item \textbf{HH:} It would help to have the process definition of an
  assessment, and a version to be applied in openETCS.
\item \textbf{MB:} The material to be assessed should be organized in
  a process-oriented way. 
\end{itemize}
 
An assessment (SJ) works like that:
\begin{enumerate}
\item Assessment of the process. If there are open points, the process
  is stopped
\item An aufit of the process is done.
\item The manufacturer sends in the full documentation. The
  phase-specific measures an techniques are assessed.
\item An audit of techniques and measures is performed.
\end{enumerate}
Along the way, the checklists are filled. Observations can have the
form of asking for an reiteration of everything (severe
omissions/violations), some improvement obligations, guidelines of how
to improve or proceed. 

\section*{Day 2 (19.02.2014)}
\label{sec:day-1-18.02.2014}

\section{Round Table (FV)}
\label{sec:round-table-fred}

\subsection{Methods and Techniques (FV)}
\label{sec:methods-techniques}

To be considered are the tables of the 50128 (in partivular A.13 and
A.14), and we have to define what goes into the quality plan and the
safety plan. 

\subsection{Interface Between WPs  (HH)}
\label{sec:interf-betw-wps}

Assign a contact person, visit the grooming meetings.

\subsection{Implementing a CENELEC Process with SCRUM (KRH)}
\label{sec:impl-cenel-proc}


\subsection{Traceability down to Goal Structuring Notation (JW)}
\label{sec:trac-down-goal}

\subsection{Safety of Code (JG)}
\label{sec:safety-code-jens}

\subsection{Train Localisation per Satellite (HM)}
\label{sec:train-local-per}

\textbf{KRH:} This is out of scope of the current project. 

\section{External Feedback (LM)}
\label{sec:external-feedback-lm}

\begin{itemize}
\item WP4 should lead the project, the V\&V has to guide the
project for it not go into wrong directions.
\item There are many different organisations working on the project.
\item \textbf{KRH:} So far, the project has been mainly concerned with
  selecting methods and tools. There is now some decision (7.1) on
  these, based on open formats. In the followup project, the
  transition to fully open-source shall be made. We have to learn the
  agile procedure. There will be just one backlog for the project for
  everybody to work on.
\item \textbf{LM:} The Subset 026 is not usable by non-experts. If the
  participants doing the modeling are not trained to expert level,
  then a requirements analysis (translating the SS~026 to a readable
  and usable form (\textbf{KRH:} Currently, it is swiss cheese with
  very big holes)) is necessary. This will be a very effort intensive
  acitivity.
\item \textbf{KRH:} With the ERSA simulator, some form of reference is
  available, for good use in the project.
\item \textbf{LM:} The tracing has to be done while constructing the
  model, this cannot be done after modeling.
\item \textbf{LM:} Personal qualification and CENELEC compliance of
  the agile model have to be assured. 
\end{itemize}

\section{Outlook Internal Assessment and Workshop Roundup (MB)}
\label{sec:outl-intern-assessm}

\begin{itemize}
\item \textbf{MB:} Next workshop in one year. \textbf{KRH:} Too
  late. make a followup workshop in the same week as the ITEA
  review (June 12). By that time, the work shall have been organized and
  progressed to some extent. Friday, June 13, 
\item \textbf{SJ, FV:} All the process documents will have to be
  prepared, and technical documentations will have to be written based
  on these documents. \textbf{KRH:} This must have been done by then,
  otherwise the project cannot work.
\item \textbf{KRH:} Is there a list of issues to be addressed or a plan?
\item \textbf{MB:} There is a high-level list of problems and
  approaches, but many details have to be added. Eg., the QA plan
  shall list all documents and activities, but its author cannot
  define the process and methods and tools---these have to be provided
  by others. In some cases, we do not even know who can provide the
  answers. 
\end{itemize}




\eod
\end{document}