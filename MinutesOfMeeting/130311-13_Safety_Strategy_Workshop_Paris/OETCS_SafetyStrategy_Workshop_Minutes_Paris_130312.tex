\documentclass[a4paper]{article}
%\documentclass[a4paper,german]{article}
\usepackage{graphicx}
\usepackage{xspace}
\usepackage{longtable}
\usepackage{array}
\usepackage{xifthen}% provides \isempty test

\pagestyle{myheadings}
\markright{openETCS \hfill {\tiny This work is licensed under a Creative Commons Attribution-ShareAlike 3.0 Unported License} \hfill}


%\usepackage{amsmath}
%\usepackage{amssymb}
% % Deutsche Silbentrennung
% \usepackage[ngerman]{babel}
% % Deutsche Umlaute
% \usepackage[ansinew]{inputenc}
% %\usepackage[latin1]{inputenc}


\setlength{\parindent}{0pt}
\setlength{\parskip}{3pt}

% editing

% Starts a new line nearly everywhere
\newcommand{\nl}{\mbox{}\\}

%Texts in a box (eg. for comments)
% Short text (no line break) 
\newcommand{\cmmnt}[1]{\framebox{#1}}
% Long text (separate lines
\newcommand{\bgcmmnt}[1]{\nl\framebox{\parbox{.95\textwidth}{#1}}\nl[2mm]}

% Macros for minutes
\newcommand{\Q}[2]{\paragraph{Question} 
	\ifthenelse{\isempty{#1}}%
    	{}% if #1 is empty
    	{by #1}% if #1 is not empty
    : #2}
\newcommand{\A}[2]{\newline{\textbf{Answer}}
	\ifthenelse{\isempty{#1}}%
    	{}% if #1 is empty
    	{by #1}% if #1 is not empty    
    : #2}
\newcommand{\C}[2]{\newline{\textbf{Comment}}
	\ifthenelse{\isempty{#1}}%
    	{}% if #1 is empty
    	{by #1}% if #1 is not empty    
    : #2}


%Uncomment for getting rid of comments in output
%\renewcommand{cmmnt}[1]{}
%\renewcommand{\bgcmmnt}[1]{}


% End of document marker
\newcommand{\eod}{\rule{\textwidth}{1pt}\nl \textit{End of Document}}





\begin{document}
\title{Workshop on Safety Strategy  \\openETCS Meeting in Paris}
\author{Marc Behrens}
\date{Version 01, 2013-03-12}

%\pagestyle{empty}

\maketitle


\section*{Document Control}

\begin{tabular}{|l|r|p{.15\textwidth}|p{.5\textwidth}|}
\hline
\multicolumn{4}{|l|}{\texttt{OETCS\_SafetyStrategy\_Workshop\_Minutes\_Paris\_130312.tex}}
\\\hline
\textbf{Version} & \textbf{Date} & \textbf{Author} & \textbf{Changes/Comment}
\\\hline
01 & 2013-03-12 & Behrens & All sections  
\\\hline
02 & 2013-03-27 & Behrens & Incorporating review comments by S.Baro on \S 2.1, 2.1.5, 2.1.6, 2.1.7  \newline Numbering refined according to the agenda.  
\\\hline
\end{tabular}

\section*{Organizational Data}

\begin{tabular}{|l|r|r|}
\hline
\textbf{Type of meeting} & \multicolumn{2}{|c|}{Face2Face}
\\\hline
\textbf{Start} & 2013-03-12 & 09:00
\\\hline
 \textbf{End} & 2013-03-12 & 17:30
\\\hline
\end{tabular}

\medskip\noindent%

\begin{tabular}{|l|r|}
\hline
\textbf{Participant} & \textbf{Organisation}
\\\hline
Armand Nachtef & CEA List \\
Baseliyos Jacob & DB \\ 
Cyril Cornu & All4tec \\
David Mentr\'{e} & Mitsubishi Electric \\
Fr\'{e}d\'{e}rique Val\'{e}e & All4tec \\
Jan Welte & TU-BS \\
Jens Gerlach & Fraunhofer FOKUS \\
Klaus-R\"udiger Hase & DB \\
Luis-Fernando Mejih &  ALSTOM BE \\
Marc Behrens & DLR \\
Marielle Petit-Doche & Systerel \\ 
Martin Schr\"{o}der & ERA \\
Merlin Pokam & AEbt \\
Pierre-Fran\c{c}ois Jauquet & ALSTOM BE\\
Ralf Pinger & Siemens \\
Renaud De Landtsheer & ALSTOM BE \\
Stan Pinte & ERTMS Solutions \\
Stephan Jagusch & AEbt \\
Sylvain Baro & SNCF 
\\\hline
\end{tabular}

\renewcommand{\contentsname}{Agenda}
\label{sec:agenda}
\tableofcontents

\section*{Results}

% Use several tabular environments to split long result lists over pages

\setlength{\extrarowheight}{1.5pt}
\begin{longtable}{|p{0.75\textwidth}|p{.025\textwidth}|p{.15\textwidth}|}
% Number (consecutive for reference)
% A (action item) OR D (decision) OR F (fact/finding) 
% Description (free text)
% responsible (for action items)
% deadline (for action items)
% header ------------------------
\hline
\textbf{Description} & \textbf{T} & \textbf{Resp.} 
%\hline
\endhead
\hline
\setcounter{section}{2}
\setcounter{subsection}{0}
\subsection{Workshop on Safety Strategy concerning Methods and Process} %2.1 

K.-R. Hase: Project Goal Priority

\begin{itemize}
%\item[Project Aims Priorization] by Klaus-R\"{u}diger
	\item [$1^{st}$] formal (or semiformal spec) specification of (step better than that what we have today) focus: formalize Subset-026 on-board unit part %$Rightarrow$ aim: 100\%
	\item [$2^{nd}$] tool's chain (to use formal spec and software generation) %$\Rightarrow$ aim: 100\% 
	\item [$3^{rd}$] running model %$\Rightarrow$ aim: 100\%
	\item [$4^{th}$] tool's chain certifiable (what is the constraint on the tool's chain/ continuously) %$\Rightarrow$ maybe not 100\% of that
	\item [$5^{th}$] (inofficial) certifiable model
\end{itemize}

\C{P.-F.Jauquet and S.Baro}{Safety cannot be considered afterwards with a low priority. The priority order is impossible to meet. Either safety should be completely discarded, or change of priority order should be done.}
\A{K.-R.Hase}{The priority is written down in the first four points.}

Validation of the specification is planned by means of tests.

K.-R. Hase: Project Goal in Steps

\begin{itemize}
	\item [1.] Formalize Prose (used to after generate code)
	\item [2a.] formal system
	\item [2.] Formal Language
	\item [3.] SW Generator 
	\item [] comment: 1.-3. are the main objectives of the project
	\item [4.] Formalize the Test cases
	\item [5.] closing the loop from 2 to 5
	\item [6.] closing the loop from 2a to 6 
	\item [7.] insert code in EVC
	\item [8.] Safety platform provided by industry
\end{itemize}

\C{F.Val\'{e}}{For safety reasons operational rules are needed}

\subsubsection{Splitting up in 2 Groups : Methods and Process} %2.1.1

& D \& F
& Klaus-R\"{u}dieger Hase
\newline
Marielle Petit-Doche 
\newline
Marc Behrens

\\\hline
\subsection{Workgroup to define user stories on Safety Strategy on Methods} %2.1.2
& F
& Renaud, Marielle, David, Pierre-Fran\c{c}ois, Jens, Sylvain, Stan, Klaus-R\"{u}diger, Luis-Fernando, Stephan
\\\hline

\subsubsection{Workgroup to define user stories on Safety Strategy on Process} %2.1.3 


For a better understanding the process shoukd be classification into:
\begin{itemize}
\item \emph{Tools} development process concerning the tool chain
\item \emph{Application} development process concerning the EVC application (application = model and executable model)
\end{itemize}

Steps that should be defined while describing the process:
\begin{enumerate}
\item Choose use case
\item Pick tools
\item Get tools implemented
\item Collect all information available
\end{enumerate}


With the fully executable functional model ambiguities should be disposed.

Different operational rules have to be respected to be applicable in different countries.

Use cases should be taken from DB and SNCF
possibilities could be:

\begin{itemize}
\item Germany: VDE 8 - Verkehrsprojekte Deutsche Einheit 
\item Netherlands:  Betuwe line
\item France: who can provide a use case?
\end{itemize}

Are BL3 use cases available?
& F
& Fr\'{e}d\'{e}rique, Armand, Cyril, Jan, Merlin, Baseliyos, Marc, Ralf
\\\hline

\subsubsection{Open debate \& decision on Methods user story} %2.1.5

SRS $\rightarrow$ Semi-Formal-Model $\rightarrow$ Striclty-Formal-Model

The model should be designed for verification.

Result:
Choose semi formal approach and strictly formal approach.

To define clear rules for strictly formal models the following should be respected:
\begin{itemize}
\item rules from strictly-formal model to the modernization
\item feedback from strictly-formal model to semi-formal model
\item (rules) to translate SRS to semi-formal model
\item coming from semi-formal to a formal model
\end{itemize}

\C{S.Baro}{The reason of the 'rules' arrow is that it is expected that the modelling of the semi-formal model needs to be made considering that the modeling of the strictly-formal model comes afterwards.}


Info:
Architectural decisions already taken at a semi-formal level

Decision affirmed on way ahead to model and Jens will sum it up and put it on Github

& D
& Jens Gerlach 
\\\hline
\setcounter{subsubsection}{5}

\subsubsection{Presentation of workshop results on Process} %2.1.6  
Results see presentation '2-1-6-User-stories\_on\_Safety-Strategy\_on\_Process\_130311.pdf'


ETCS Basleine 3 use cases are not yet in operation.

\C{M.Pokam}{The need of operational scenarios for the process containing operational rules was identified during the workshop.}
\Q{}{Which operating rules are we talking about?}
\A{K.-R.Hase}{Betuweroute \& VDE 8 are proposed within the FPP}

\Q{Session Participants}{open Point: Do we need Aim 4 and if yes who will define the safety strategy 
If Safety considered you have to put it on top  --> Is safety considered?}

& D
& Merlin Pokam
\\\hline
\subsubsection{Open debate \& decision on Process user story} %2.1.7

Project has to deliver otherwise the project is failed

We can not guarantee that we have a certifiable toolchain/ model
But we can ...

\Q{M.Behrens}{Who identifies the subset to make an assessment on it?}
Proposition: Mid of April (Workshop) - What is the requirement on which we will focus the safety demonstration of the tool's chain

\C{S.Baro \& P.-F. Jauquet}{Decision: Model as much as possible all of the OBU and then try to do the whole safety activities on a small part of the subset.For which the tool chain and process are compliant.}

\C{P.-F. Jauquet}{WP3 takes a small part of the model out of the tool evaluation project to evaluate the safety model.}

\paragraph{Focus:} Only the semi-formal model and the strictly-formal should be safety certifiable.


\paragraph{Decision} Safety proof will be done on semi-/ strictly- formal model.
\paragraph{Decision} Code generation is taken out of the safety strategy. No T3 Code generator is in focus.

& F
& Cyril Cornu
\\\hline
\subsection{Follow-up user stories}
\subsubsection{Usage of openETCS-tools for development of ETCS on-board} %2.2.1 

Siemens would like to integrate to whatever comes out of openETCS.

Formalization means lots of manual effort.

The right hand of the V-cycle can be used for conformance testing.

Tool qualification T1 or T2 should be needed for right hand V-cycle (CENELEC EN50128 CH6.7)

Siemens would like to contribute to SIL-4 certifiable code.

	Risk \& Hazard analysis, competence management needed in parallel for SIL-4 developement.

For Siemens the best contribution would be to having the openETCS results at the left part of the V-cycle.

Agreement on the Test-API
	Testing interfaces for testing (e.g. DMI)
	Chance to agree on a Test-API : Siemens could contribute there.


$\Rightarrow$ Left hand side of the V: executable SIL4 is very ambitous

$\Rightarrow$ Tool it makes sense to start a light tool.

& F
& Ralf Pinger
\\\hline
\subsubsection{V\&V Strategy and Scrum} %2.2.2
User Story of the Verificator is presented by M. Behrens

Each user story has to fulfil the acceptance criteria and be testable.
& F
& Marc Behrens
\\\hline
\subsubsection{Verification of code and API} %2.2.3 

Results see presentation.

It is important to investigate several properties to be formalized.
Properties may come directly from the SRS or the API.
	e.g. integers are in a certain bounds, pointer always allocated, e.g. all integrer operation in the implementation never overflows, time constraints

Different properties to be verified to what extend they can contribute to a high level of assurance.

Pie diagrams in green and red within the model says: Some properties i.e. real time property of the executable code can probably not be mathematically verified.
& F
& Jens Gerlach 
\\\hline
\subsubsection{Project-, QA-Plan and Scrum} %2.2.4
Questions and answers see presentation.

Presenting of project plan.

Decision was voted to call a PCC meeting. Within the next PCC meeting 

Backlog is a tasklist and the product-owner of the backlog are the WP leader or Taskleader.

For each task we have a backlog.

The tool Jira with Greenhopper is currently in evaluation (for distributed Scrum teams).
& F
& Baseliyos Jacob
\\\hline
\subsubsection{Conclusion} %2.2.5
The following decisions were agreed on:

\textbf{4 goals} for the project have been defined:
\begin{description}
	\item [$1^{st}$ priority:] semi-formal (or formal) specification (step 		better than that what we have today) with the focus on: formalizing subset-026 on-board unit
		\newline
		aim: 100\%
	\item [$2^{nd}$ priority] tool's chain (to use formal spec and software generation)
		\newline
		aim: 100\% 
	\item [$3^{rd}$ priority:] running model 
		\newline
		aim: 100\%
	\item [$4^{th}$ priority:] tool's chain certifiable (what is the constraint on the tool's chain/ continuity)
		\newline
		aim: for validation of the safety strategy
\end{description}

\paragraph{Method} A development method integrating the following artifacts is foreseen:
\begin{itemize}
	\item SRS
	\item Semi-Formal Model
	\item Strictly-Formal Model
	\item Running Software Model	
\end{itemize}


\paragraph{Process} As a basis to evaluate the results for functional as well as safety reasons \emph{Use Cases on Operational Basis} should be identified. It was agreed that within the project \emph{safety does not touch code generation}. A small subset should be identified for the safety case. 

\paragraph{Qualification of openETCS tools} is recommended to be started on the \emph{Right Side of the V-Cycle} (T1, T2). After this qualification has proven in use the qualification for the \emph{Left Side of the V-Cycle} is advised (depending on the 4th priority) (T3).


\begin{description}
	\item [V\&V SCRUM process]
		\subitem [based on user story]
		\subitem The user story, respecting the acceptance criteria, has to be testable within a defined timeslot 
	\item [API Verification]
		\subitem [focus on the] different properties to be verified to what extend they can contribute to a high level of assurance
	\item [project plan]
		\subitem [Project Plan]proposed
		\subitem [PCC] meeting is voted on, the PCC will be planned around next month's in Munich 
		\subitem [Releaseplan] is to be defined by the productowner e.g. Task leader, WP-L
		\subitem [QA-Plan] is worked on currently 
		\subitem [scrum] more scrum training for people is identified
	\item [Safety meeting] is foreseen within the next weeks.
\end{description}
 
& D 
& Marc Behrens
\\\hline
\end{longtable}

\bgcmmnt{\textbf{T} for type of item:
\begin{description}
\item[A] action item
\item[D] decision
\item[F] fact / finding
\end{description}}



\section*{Notes}

There may be more elaorate formats for protocols. This format lacks references to ITEA~2 so far.

% Optional for additional free text
  
\eod


 

\end{document}