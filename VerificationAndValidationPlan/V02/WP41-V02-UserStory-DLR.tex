%\documentclass{article}

%\title{Proposal for a Subsection Describing a VnV User Story in the
%  Verification Plan:\\Instantiated for a DLR Verification Activity}
%\author{Hardi Hungar}
%\date{July 16, 2014}

%\newcommand{\tbi}[1]{$<$\textit{#1}$>$}


%\begin{document}
%\maketitle

%\begin{abstract}
%  This document presents an the V\&V user story of the DLR for
%  Verification Level~2, to exemplify how the provided template can be
%  instantiated.
%\end{abstract}

%\subsection{Applying RT-Tester to a
% Model of a Component Handling the Acknowledgment  of a Level Transition Order }

This section describes the verification activity of the DLR during
Verification Level 2. The object of verification is 
a model of a component handling the acknowledgment  of a level
transition order by the driver. This component shall trigger braking if
the driver does not acknowledge the level change as required. The goal
of the activity is mainly to evaluate the RT Tester tool as a
component of the secondary tool chain. The verification object itself
will be created as part of the activity. It is intended to contribute
it to WP~3 (Modelling) as a byproduct of this activity.


\paragraph{Object of verification}
The object of verification will be available in the modeling
repository on GitHub after it has reached a certain degree of
maturity. It is intended to constitute a \emph{module design
  specification}, i.e., a specification to be implemented using SCADE
to generate code in a subsequent development step. It will be
developed as a SysML model with C annotations to make it
executable. It describes how the OBU monitors potentially
necessary driver acknowledgments after a \texttt{Level Transition
  Order} (LTO) has been received.  The term LTO Monitor will be used
to refer to the component in the ensuing description.

\paragraph{Available specification}
Section~5.10.4 of Subset~026 is taken as functional
specification. Details like the interface objects of the LTO Monitor
and functionalities of other OBU components which the monitor relies
on will be defined based on a general understanding of the workings of the
OBU and a discussion with project contributors (lacking a SW
architecture and component design so far).

\paragraph{Methods and Means}

The main tools employed in this activity are Papyrus for SysML
modeling (Primary toolchain, deliverable D7.1) and RT Tester,
Sec.~\ref{sec:RTTester}. For modeling, we adhere to the language
restrictions of the RT-Tester manual [\emph{add citation}] in order to
be able to use the model as a \emph{simulation}. This way, we can
compile the LTO Monitor and run tests on it. Tests are to be generated
from \emph{test models} focusing on parts of the monitor
functionality.  

\paragraph{Results to be achieved}

The intention is get a feeling for the capabilities of RT-Tester to be
able to assess its potential for implementing the openETCS
workflow. Findings and conclusions shall be entered into a report, to
be discussed with U Bremen (who contribute the RT Tester) and
potential users of it. 

\paragraph{Timeline}

\begin{itemize}
\item Modeling the LTO Monitor, writing test models and applying the tests
to the monitor shall be done during the second level of
verification. The activity shall be completed by October 2014.
\item There are no results so far as the activity has started with
Verification Level~2.
\item No plans for continuing the activity in Verification~Level~3 have been
made. 
\end{itemize}






\paragraph{Maturity Classification}

The tools applied have the following TRLs (Technology Readiness
Levels):
\begin{description}
\item[Papyrus:] TRL~4 (tentative estimation). Papyrus SysML modeling
  has been employed by the developers and potentially also in an
  industrial context, but not in the form it is done here. WP~7 will
  be in a better position to judge that.
\item[RT Tester:] TRL between 4 and 8. According to U Bremen (Jan
  Peleska), RT Tester has been assessed for use in the development
  regulated by relevant standards. More details on that will be
  provided by U Bremen. Wrt.\ the current usage context (Papyrus
  SysML), the evidence would have to be evaluated to assign a similar
  TRL (higher than 4).
\end{description}


This activity shall not comply to the requirements of a SIL~4
development, as it is intended to evaluate the tools. A positive
outcome of the evaluation would contribute to an argument that RT
Tester is appropriate to perform a part of the verification of design
specifications which take the form of executable models. This relies on:
\begin{enumerate}
\item A potentially high maturity of RT Tester (its
qualification)
\item The fact that systematic dynamic testing with high
coverage is an element of an approved combination of techniques for
code verification. And that executable models would be adequately
treated similar to code.
\end{enumerate}
This is of course a preliminary statement to be substantiated at the
end of the activity.  

%\eod

%\end{document}