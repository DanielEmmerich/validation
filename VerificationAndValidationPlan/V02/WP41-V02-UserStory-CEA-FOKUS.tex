% \subsection{Formal Verification at Software Level}
% \label{sec:form-verif-soft-openETCS}
This section describes CEA LIST's and Fraunhofer FOKUS' plans regarding the use
of formal methods to assess properties at the C code level. 
Section~\ref{sec:Frama-C} above describes the main plug-ins of the Frama-C
tool suite that are envisaged for that, while the theoretical background is
summarized in sections~\ref{sec:Abstract Interpretation}
and~\ref{sec:deduct-verif}. Namely, two main categories of properties
can be dealt with. First, we can focus on functional properties, that is
establishing that a given function is conforming to its (formal) specification.
Second, it is also possible to analyze a whole application to check the absence
of potential run-time errors (arithmetic overflows, division by 0, invalid
dereference of pointers, buffer overflow, use of uninitialized variables,
undefined order of evaluation, ...). A case study partly based on previous
experiments is developed further in OpenETCS and has been presented
in~\cite{Gerlach.2013}. Existing code from OpenETCS partners,
namely Siemens and ERSA has also been identified has a good target for
such activities.