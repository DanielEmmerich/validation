
% \chapter{\VV Strategy for the Project openETCS}
% \label{sec:vv-strategy-project}

\paragraph{Scope of \VV in openETCS}

The project will only perform part of the development, and thus also
only a part of the V\&V activities. These need to be defined and
planned. The overall approach shall be to try out the different
constituents on and in representative samples, so that the
realisability of the \vv strategy for a full development can
covincingly be demonstrated.

Besides the usual purpose of \vv activities, namely evaluating and
proving the suitability of design artifacts, V\&V in openETCS will
also generate information on the suitability of the methods and tools
employed. For that purpose, a format for describing methods and tools
to be used in V\&V and one for summarizing the findings about the
suitability are defined.

\paragraph{Implementing SCRUM}
The overall project approach is to organise activities in an agile
way, following the ideas behind the SCRUM development
method. This has two main aspects for V\& V (in the following text,
\emph{verification} is used to also include validation):

\begin{enumerate}
\item \label{SCRUM-Artifacts} Artifacts are produced in iterations and
  expected to be verified iteratively.
\item \label{SCRUM-Split} Verification activities are split over
several iteration, where each should produce some useful result.
\end{enumerate}

Adressing (\ref{SCRUM-Artifacts}), it is recommended to not attempt to
derive a final verdict, except in cases where a set of requirements
are claimed to be covered completely. Preliminary models or code parts
can serve to evaluate tools and methods, and to set up verification
environments, to be able to complete the verification when the
artifact is a more final state. Preliminary artifacts should be
assessed for their aptness for the intended v\&v. Feedback on
deficiencies will help to accelerate later activities.

Concerning (\ref{SCRUM-Split}), an adequate procedure is to include a
review of the verification in each iteration. This corresponds to the
principle of only permit ``tested code'' into the results of each
SCRUM phase. This review should best be performed by a project partner
different from the one who performed the verification.

Further practices to do verification during an ongoing development
with several parameter not set, include: 
\begin{enumerate}
\item Write an object of verification if necessary for evaluation purposes
\item Construct specifications from available material
\item Try to fit those into a general picture of the development (process)
\item Select methods and tools prudently (keep FLOSS and CENELEC in mind)
\end{enumerate}



\paragraph{Cooperation with Other WPs}

The \vv has to be performed in cooperation with WP~3, which
produces DAS2Vs (models and code), and with WP~7, where methods
and tools are defined and developed. 

To exchange information with WP~3, formats are needed for collecting
information about DAS2Vs (V\&V tasks) and for giving back information
about the results of V\&V activities. Similarly, with WP~7
communication shall use formats to describe V\&V methods and tools
(input from WP~7) and the results of evaluations of V\&V methods and
tools.