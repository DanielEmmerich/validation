% \chapter{Verification Plan for the Project openETCS}
% \label{sec:verification-plan-project}

\section{Verification Overview}
\label{sec:verif-overv}
Describe the organization, schedule, resources, responsibilities,
tools, techniques, and methodologies to be deployed in order to
perform the verification activities.


\subsection{Organisation}
Define the relationship of verification to other efforts such as
development, project management, quality assurance, and configuration
management. Define the lines of communication within the verification
effort, the authority for resolving issues, and the authority for
approving verification deliverables.

Organisation: a format for describing design artifacts subject to V\&V,
  and a feedback format for the findings during V\&V.

\subsection{Schedule}
The schedule summarizes the various verification tasks and their
relationship to the overall openETCS project.  It describes the
project life cycle and project milestones including completion dates.
Summarize the schedule of verification tasks and how verification
results provide feedback to the whole openETCS process to support
overall project management functions.  The objective of this section
is to define an orderly flow of material between project activities
and verification tasks.

According to the Description of Work of WP~4 \cite{WP4DoW}, the verification
activities will be structured into three main phases: 
\begin{enumerate}
\item First Level: Verification of prototypical system and API model
  and prototypical code,
\item Second Level: Verification of system model, functional API
  prototype model, code architecture and system API prototype
\item Third Level: Verification of final system and API model, final
  code and the functional API model
\end{enumerate}







\section{Verification Activities---User Stories}
\label{sec:verif-activ-user}

The term ``User Story'' as used in openETCS stands for any kind
application of tools, not just for the end user application of the
system (EVC software) which is to be developed. This section shall
describe such ``user stories'' of verifiers, i.e., it
shall describe where which method or tool is to going to be applied to
what artifact(s) (DAS2Vs). It thus shall tell coherently the
\textbf{story} of verification activities. Later (sub)sections provide
the organisational detail: \textbf{When} (Timeline,
Sec.~\ref{sec:verif-activ-timel}) and \textbf{Contribution} (which of
the verification obligations from Sec.~\ref{sec:verif-full-devel} are
tackled by what approach.

\bgcmmnt{the template for a user story should be included. It needs to
be updated: Verification Levels 1 and 2 should be covered. One may
include the old plan for Level 1, and mark changes appropriately.}


\subsection{Reviews and Inspections}
\label{sec:reviews-inspec-openETCS}

%\subsection{Reviews and Inspections}
%\label{sec:reviews-inspec-openETCS}

\paragraph{Reviews}
In the openETCS project
all written documents, specifications, models and code can be
reviewed. It is also important to include all documents concerned with
the creation and delivery of the openETCS product. This means that
strategies, plans, approaches, operation and maintenance manual, user
guides, the contract that will initiate the work should all be
reviewed in a structured way. 

Information about the reviews planned within openETCS is given in the
QA plan \cite{QAplan}.

\paragraph{Inspections}
Inspection shall be applied to design artifacts whose correctness
is important for the demonstration of the project results, and also
to validate results of innovative \vv methods and/or tools.

\subsection{Software Architecture Analysis Method (SAAM)}
\label{sec:saam-openETCS}

%\subsection{Software Architecture Analysis Method (SAAM)}
%\label{sec:saam-openETCS}
Potential targets for SAAM are parts and instances the SFM 
(semi-formal model) of the software, to ascertain the 
viability of the software architecture decision. The scenarios 
with which the analysis shall be performed are to be developed
from the operator requirements.

\subsection{Architecture Tradeoff Analysis Method (ATAM)}	
\label{sec:atam-openETCS}

%\subsection{Architecture Tradeoff Analysis Method (ATAM)}	
%\label{sec:atam-openETCS}
ATAM may be applied to the SW architecture.

\subsection{Formal Verification at Software Level}
\label{sec:form-verif-soft-openETCS}

% \subsection{Formal Verification at Software Level}
% \label{sec:form-verif-soft-openETCS}
This section describes CEA LIST's and Fraunhofer FOKUS' plans regarding the use
of formal methods to assess properties at the C code level. 
Section~\ref{sec:Frama-C} above describes the main plug-ins of the Frama-C
tool suite that are envisaged for that, while the theoretical background is
summarized in sections~\ref{sec:Abstract Interpretation}
and~\ref{sec:deduct-verif}. Namely, two main categories of properties
can be dealt with. First, we can focus on functional properties, that is
establishing that a given function is conforming to its (formal) specification.
Second, it is also possible to analyze a whole application to check the absence
of potential run-time errors (arithmetic overflows, division by 0, invalid
dereference of pointers, buffer overflow, use of uninitialized variables,
undefined order of evaluation, ...). A case study partly based on previous
experiments is developed further in OpenETCS and has been presented
in~\cite{Gerlach.2013}. Existing code from OpenETCS partners,
namely Siemens and ERSA has also been identified has a good target for
such activities.

\subsection{Applying RT-Tester to a
  Model of a Component Handling the Acknowledgment  of a Level Transition Order }

%\documentclass{article}

%\title{Proposal for a Subsection Describing a VnV User Story in the
%  Verification Plan:\\Instantiated for a DLR Verification Activity}
%\author{Hardi Hungar}
%\date{July 16, 2014}

%\newcommand{\tbi}[1]{$<$\textit{#1}$>$}


%\begin{document}
%\maketitle

%\begin{abstract}
%  This document presents an the V\&V user story of the DLR for
%  Verification Level~2, to exemplify how the provided template can be
%  instantiated.
%\end{abstract}

%\subsection{Applying RT-Tester to a
% Model of a Component Handling the Acknowledgment  of a Level Transition Order }

This section describes the verification activity of the DLR during
Verification Level 2. The object of verification is 
a model of a component handling the acknowledgment  of a level
transition order by the driver. This component shall trigger braking if
the driver does not acknowledge the level change as required. The goal
of the activity is mainly to evaluate the RT Tester tool as a
component of the secondary tool chain. The verification object itself
will be created as part of the activity. It is intended to contribute
it to WP~3 (Modelling) as a byproduct of this activity.


\paragraph{Object of verification}
The object of verification will be available in the modeling
repository on GitHub after it has reached a certain degree of
maturity. It is intended to constitute a \emph{module design
  specification}, i.e., a specification to be implemented using SCADE
to generate code in a subsequent development step. It will be
developed as a SysML model with C annotations to make it
executable. It describes how the OBU monitors potentially
necessary driver acknowledgments after a \texttt{Level Transition
  Order} (LTO) has been received.  The term LTO Monitor will be used
to refer to the component in the ensuing description.

\paragraph{Available specification}
Section~5.10.4 of Subset~026 is taken as functional
specification. Details like the interface objects of the LTO Monitor
and functionalities of other OBU components which the monitor relies
on will be defined based on a general understanding of the workings of the
OBU and a discussion with project contributors (lacking a SW
architecture and component design so far).

\paragraph{Methods and Means}

The main tools employed in this activity are Papyrus for SysML
modeling (Primary toolchain, deliverable D7.1) and RT Tester,
Sec.~\ref{sec:RTTester}. For modeling, we adhere to the language
restrictions of the RT-Tester manual [\emph{add citation}] in order to
be able to use the model as a \emph{simulation}. This way, we can
compile the LTO Monitor and run tests on it. Tests are to be generated
from \emph{test models} focusing on parts of the monitor
functionality.  

\paragraph{Results to be achieved}

The intention is get a feeling for the capabilities of RT-Tester to be
able to assess its potential for implementing the openETCS
workflow. Findings and conclusions shall be entered into a report, to
be discussed with U Bremen (who contribute the RT Tester) and
potential users of it. 

\paragraph{Timeline}

\begin{itemize}
\item Modeling the LTO Monitor, writing test models and applying the tests
to the monitor shall be done during the second level of
verification. The activity shall be completed by October 2014.
\item There are no results so far as the activity has started with
Verification Level~2.
\item No plans for continuing the activity in Verification~Level~3 have been
made. 
\end{itemize}






\paragraph{Maturity Classification}

The tools applied have the following TRLs (Technology Readiness
Levels):
\begin{description}
\item[Papyrus:] TRL~4 (tentative estimation). Papyrus SysML modeling
  has been employed by the developers and potentially also in an
  industrial context, but not in the form it is done here. WP~7 will
  be in a better position to judge that.
\item[RT Tester:] TRL between 4 and 8. According to U Bremen (Jan
  Peleska), RT Tester has been assessed for use in the development
  regulated by relevant standards. More details on that will be
  provided by U Bremen. Wrt.\ the current usage context (Papyrus
  SysML), the evidence would have to be evaluated to assign a similar
  TRL (higher than 4).
\end{description}


This activity shall not comply to the requirements of a SIL~4
development, as it is intended to evaluate the tools. A positive
outcome of the evaluation would contribute to an argument that RT
Tester is appropriate to perform a part of the verification of design
specifications which take the form of executable models. This relies on:
\begin{enumerate}
\item A potentially high maturity of RT Tester (its
qualification)
\item The fact that systematic dynamic testing with high
coverage is an element of an approved combination of techniques for
code verification. And that executable models would be adequately
treated similar to code.
\end{enumerate}
This is of course a preliminary statement to be substantiated at the
end of the activity.  

%\eod

%\end{document}

\subsection{Formal Model Verification (TWT)}
\label{sec:real-time-TA-openETCS}

%\subsection{Formal Model Verification (TWT)}
%\label{sec:real-time-TA-openETCS}

This section describes TWT's plans regarding the formal verification
of the system model. While TWT's original approach has been focused
primarily on real-time aspects by employing timed automata and the
UPPAAL tool as described in Section~\ref{sct:twt:descrTA} (page
\pageref{sct:twt:descrTA}) our activities are now addressing formal
model verification on a broader scale.

TWT's work will be based on the following action items:
\begin{enumerate}
\item We will investigate the suitability of the open source toolkit
  CPN Tools\cite{CPNTools} for formal verfication (model checking) and
  simulation. The evaluation of CPN Tools will be aligned with the
  secondary toolchain assessment in WP7. A first step in the tool
  evaluation is the construction of an example model (currently
  ``Start of Mission'') from the ETCS specification.
  \item We will analyze whether/how UML/SysML statecharts and possibly
activity diagrams can be transformed to colored Petri nets to permit
their simulation and the formal verification of requirements.
  \item We will analyse the expressive power of the CPN Tools property
  language and whether the modeling formalism can replace UPPAAL for
  modelling and verifying timed aspects of ETCS components.
  \item We will provide feedback regarding ambiguities,
  inconsistencies and errors in the current ETCS standard based on our
  formalisation and the analysis of other models based on our
  approach.
  \item As the University of Braunschweig has experience with formal
  Petri net modelling, we will align our efforts with them.
  \item We will investigate, how our approach can be integrated in the
Eclipse environment and develop such integration if feasible with our
efforts.
  \item We plan to publish our results on action item 1.
\end{enumerate}

Please note that the above list may be subject to future change.



\subsection{Verification with Model-Based Simulation using SystemC (TWT, URO)}
\label{sec:model-based-sim-openETCS}
%\subsection{Verification with Model-Based Simulation using SystemC (TWT, URO)}
%\label{sec:model-based-sim-openETCS}

In Section~\ref{sct:uro:systemc} (page \pageref{sct:uro:systemc}) the
basics of SystemC and the SysML/SystemC joint approach have been
described. TWT and the University of Rostock (URO) will work on this
concept based on the following action items:

\begin{enumerate}
  \item TWT will analyse methods for generating SystemC code from
  SysML models. In first investigations the Acceleo tool (Eclipse
  plugin, based on OMG standard) seems to be a promising candidate for
  model-to-text transformation.
  \item URO will build a modular and executable SystemC model for
  braking curves that is suitable for real-time simulation. An
  accompanying high-level SysML model will be constructed as well and
  will be means to test the transformation methods to be developed in
  the context of action item 1.
  \item TWT and URO plan to investigate which other parts of the ETCS
  specification can benefit from real-time simulation and build models
  accordingly.
  \item URO will investigate whether performance analysis based on the
  underlying hardware system is feasible within the openETCS
  project. This will allow to scale the hardware resources of the OBU
  system accordingly.
  \item In addition, using the results from action item 1, TWT and URO
    plan to transform existing SysML models (to be developed in WP3)
    to SystemC for real-time simulation.
  \item Evaluation (and possibly implementation) of an Eclipse
  integration
\end{enumerate}

Please note that the above list may be subject to future change.


\subsection{System Integration  Testing (Uni Bremen/DLR)}
%\subsection{System Integration  Testing (Uni Bremen/DLR)}

Section \ref{subsec:mbt} describes the main objective of system
integration testing. Uni Bremen will focus on the following items~:
\begin{enumerate}
\item Create a test model in SysML. From the model evaluation
  activities, the management of the radio communication is already
  available. To cover different aspect of the specification, the
  ceiling speed monitoring model will be also provided.
\item Generate test cases according to the defined interface given by
  DLR. (RT-Tester) 
\item Provide simulation environment for the
  track-to-train simulation (including braking curves/speed
  profiles)\footnote{ OpenETCS system testing for the EVC on-board
    computer requires test execution in real physical time and also
    track layout with realistic speed profiles.  We also want to
    contribute to the track and Speed Simulation by automatically
    generates ``relevant'' layout for OpenETCS.}  along routes used
  for testing
\item Study the automatic generation of these track layouts and speed
  simulations; if feasible, implement a generator and integrated it in
  the DLR laboratory environment
\item Set up a test environments for
  \begin{itemize}  
  \item Hardware-in-the-loop Testing within DLR laboratory.
  \item Software-in-loop testing with code provided by SCADE (Siemens)
  \end{itemize}
\end{enumerate}


\paragraph{Track simulation}
OpenETCS system testing for the EVC on-board computer requires test
execution in real physical time and also track layout with realistic
speed profiles.  We also want to contribute to the track and Speed
Simulation by automatically generates ``relevant'' layout for OpenETCS.


\paragraph{Test cases generation}
We also plan different activities to ensure the pertinence of our test cases.
\begin{enumerate}
\item Check the test model (SysML) : RTT-BMC or other model checkers
\item Add relevant LTL properties if needed
\item Test case analysis by 
  \begin{itemize}
    \item Structural coverage
    \item Requirement coverage
    \item Mutation coverage 
    \item Data coverage
  \end{itemize}
\end{enumerate}

\paragraph{Test cases analysis -- comparison to Subset 76}
\begin{enumerate}
\item Provide techniques and Howto describing how test cases from
  Subset 76 can be executed in the RT-Tester environment, either as SW
  integration test or as HiL test in the DLR simulation environment
\item Create new set of test cases for the ceiling speed monitoring
  (As far as we know,they do not yet exist in Subset 76)
\item Compare new test cases created by RT-Tester to new test cases
  for ceiling speed monitoring provided by ERTMS standardization
  group, as soon as available; suggest improvements for the Subset 76
  test cases. 
 \end{enumerate}

%\paragraph{Object Code Verification}
%---> please suggest something!!!

\paragraph{Exchange Formats}
Test models represented in XMI/Ecore are used as SysML test modeling
standard. RT-Tester model parsers are extended to cope with this
format.

Test procedures will be represented in a general abstract syntax
format, so that procedures generated by RT-tester can be run on any
test execution platform.

Test results (test execution logs) will be represented in a general
format, so that exchange of test results between tools (for example,
for simulation purposes) becomes possible.



A first part of these activities have already been done for model of
radio management communication (see the model-evaluation for
\href{https://github.com/openETCS/model-evaluation/blob/master/model/EA-SysML/new_version/doc/ea_sysml_report.pdf}{EA/RT-tester}).


\subsection{Model Verification by applying the openETCS
  Verification Tool Chain (Siemens)}
% \subsection{Model Verification by applying the openETCS
%   Verification Tool Chain (Siemens)}

Siemens will focus the activities on the verification of its
contributions to the openETCS work packages 3 and 7. Currently
available is the MoRC model provided for the primary tool chain for WP
7. This modeling will be continued for WP 3. The existing formal model
will be complemented by a semi-formal model on top. Both, the
semi-formal as well as the formal model need to be verified in WP 4
and will serve as verification objects.

The intention is to apply the preferred and most valuable methods and
parts of the openETCS verification tool chain to the models and by
giving feedback into the openETCS project contribute to increase their
useability and adequacy.

In more detail, these activities are planned: 

\begin{itemize}
\item Support the integration of the MoRC model into the openETCS test
  environment / DLR laboratory.
\item Support the integration of the MoRC model into the RT-Tester
  environment, provided by Uni Bremen.
\item Model based test case generation and execution by using the
  RT-Tester environment provided by Uni Bremen.
\item Determination of the structural test coverage on the model.
\item Determination of the requirements coverage regarding subset-026
  documents, SSRS, implementation model, test model, test case and
  test execution issues.
\item Determine the feasibility of proving safety properties for the
  model.
\end{itemize}


\section{Verification Activities---Timeline}
\label{sec:verif-activ-timel}

This section lists per partner/activity in which of the project
phases according to Sec.~\ref{sec:verif-activ-timel} (first, second or
third level of verification) a particular activity is planned. In the
first version of the V\&V plan, only the first level needs to be detailed.


\subsection{First Level of Verification}
\label{sec:first-level-verif}

This section gives only a short description, which should refer to an
activity detailed earlier. Most probably the material will be
organised in a table.

\paragraph{TWT}
TWT will continue with the analysis of methods for generating SystemC
code from SysML models. In first investigations the Acceleo tool
(Eclipse plugin, based on OMG standard) seems to be a promising
candidate for model-to-text transformation. The approach will be
aligned with URO and their SystemC model of the braking curves.  See
also Sct.~\ref{sec:model-based-sim-openETCS}.

In addition, TWT will start with the analysis on whether/how UML/SysML
statecharts can be transformed to timed automata in a sensible manner
while retaining as much structural information as possible. Timed
automata provide a means for model checking real-time properties of
systems. See also Sct.~\ref{sec:real-time-TA-openETCS}.

\paragraph{URO}
URO will continue their work on a modular and executable SystemC model
for braking curves suitable for real-time simulation. An accompanying
high-level SysML model will be constructed as well and aligned with
TWT's activities on SysML $\rightarrow$ SystemC code generation.  See
also Sct.~\ref{sec:model-based-sim-openETCS}.

\paragraph{Uni. Bremen}
Uni Bremen will start by developing and setting a simulation
environment for the test of the EVC. The test models from the
model-evaluation will be completed and the breaking curves will be
added.  The set of tests generated will be then compared to the
available ones of SUBSET-076.


\subsection{Second Level of Verification}
\label{sec:secon-level-verif}

\subsection{Third Level of Verification}
\label{sec:third-level-verif}


\section{Verification Activities---Process View}
\label{sec:verif-activ-proce}
This section provides the detailed plan for the verification tasks
throughout the openETCS project life cycle. It summarizes the
activities performed by the project partners in relating them to the
overall definition of verification activities in
Sec.~\ref{sec:verif-full-devel}. 

\section{Verification Reporting}
This section describes how the results of implementing the
Verification Plan will be documented.  Verification reporting will
occur throughout the software life cycle.  The content, format, and
timing of all verification reports shall be specified in this section.

\bgcmmnt{Here we need the template for reporting.}

\nthng{\textit{This subsection has to be revised to fit the restrictions of implementability within the project.}}

The following reports will be generated during the verification process:
\begin{itemize}
\item \textbf{Anomaly reports:} 
\item \textbf{Phase Summary Verification reports:} 
\item \textbf{Final report:}
\end{itemize}

The structure of the Verification report is already defined in the
\ref{sec:struct-verif-report} section of this document 

\section{Administrative procedures}
This section identifies the existing administrative procedures that
are to be implemented as part of the Verification Plan. 
Verification efforts consist of both management and technical tasks.
Furthermore, it is the task of the SQA team to monitor whether the
procedures as defined in the management plans ([QAplan], [SCMP],
[Review and Revision processes]) are followed. 

\subsection{Problem Report}
The problem reporting procedure is described within the document
Change/Problem Management Process.

Any problem, failure and error encountered during the review
activities (QA. Verification, Validation, Assessment) planned in the
software development life-cycle, problems reported by users and
customers as well as change requests initiated by any of the system
stakeholders will be reported and managed following the Change/Problem
Management Process detailed in
\href{https://github.com/openETCS/governance/tree/master/Change-Problem%20Process}{[governance]} and through the Change/Problem Management Tool.

\subsection{Task Iteration Process}
Any change in the requirements (system, sub-systems, sw or components)
require repeated verification and validation activities. 

Once the change is accepted following the change/problem management
procedure, the phases and items affected by it must be
evaluated. These tests will be redesigned to reflect the change in the
requirement and will be executed again. 

In turn, a new analysis of the Software Integrity Level will involve
the analysis of the activities requirements and documentation
presented by the EN50128 standard and include such activities in the
SVVP if necessary. 

\subsection{Deviation Process}
The Quality Manager will be informed in the case of detection of a
deviation regarding Verification Plan. In addition, he/she also be
informed if it is deemed necessary by an amendment to the Plan,
whether or not motivated by a deviation 

The Quality Manager will report such incidents to the Project Managers
and with whom shall act appropriately. All persons listed in the
Responsibilities section (Sec.~\ref{sec:vv-responsibilities}) shall be
informed of a change in the Verification Plan

\subsection{Control Procedure}
Control procedures are specified in the Configuration Management Plan [SCMP]
