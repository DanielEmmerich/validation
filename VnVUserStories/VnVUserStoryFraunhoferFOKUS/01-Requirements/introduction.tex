
\section{Introduction}
\label{sec:introduction}

The original, high-level description of the \bitwalker software is given here:

\begin{quote}
 
Task of the Bitwalker
 
The Bitwalker shall sequentially read a bit field and convert it into a natural number.
Conversely, the Bitwalker shall read a natural number and write it to a bit field.
 
Description of the task
 
A bit oriented field in memory is given.

With a byte oriented representation in memory, the most significant bit
would have the index 0 in Big-endian notation.
The bit index increases monotonously by 1.
Natural numbers that are given in Big-endian notation shall be read or written.
The range of natural numbers is limited by the width of the bit field.

Two access methods shall be provided:
\begin{itemize}
\item  Random read/write access to the bit field. 
\item  Sequential read/write access to the bit field starting at a given index.
\end{itemize}

Each read/write operation shall ``consume'' the bits that have been accessed.
The access functions shall be written in plain C and shall be reentrant.
 
Context of the Bitwalker:
 
ETCS uses telegrams in Big-endian notation to communicate between vehicle and track.
In the scope of OpenETCS a generator shall be developed that
automatically converts the telegram specification into an encoder/decoder.
The Bitwalker will be used here as an auxiliary function.
\end{quote}

While this description provides basic information about the context
of \bitwalker and about the intentions of the designer there are 
many open questions:

\begin{itemize}
\item What are the exact names and signatures of the functions?
\item Are there any error conditions for these functions?
\item What does ``consumption of bits'' mean?
\item What does \emph{reentrant} exactly mean?
\end{itemize}


The \bitwalker software can be considered to consist of a \emph{public} part
(sequential access functions)
and a \emph{private} part (random access functions).

\begin{itemize}
\item
The public part consists of the \isoc data type \bitwalkertype and several \isoc functions
that can be used to manipulate objects this type.

\item
The private part consists of the functions \peek and \poke
that implement the core functionality of \bitwalker.
\end{itemize}

Before we informally specify \bitwalker
we introduce some auxiliary concepts and formulate general assumptions.
We would also like to point out the following:
When we speak in this document of \emph{integers},
then we refer to the infinite set of mathematical
integers $\{\ldots, -1, 0, 1, \ldots\}$
and not to one of the many finite representation provided by the type system of \isoc.

This distinction is important because it usually makes more sense
to describe the functionality of a piece of software in a more
abstract way.
In a later step the realities of specific the \isoc type system
have to be taken into account.

