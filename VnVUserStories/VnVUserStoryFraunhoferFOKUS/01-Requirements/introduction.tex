
\chapter{Introduction}
\label{sec:introduction}

The original, high-level description of the \bitwalker software is given here (in German):

\selectlanguage{german}

\begin{quote}
 
Aufgabe des Bitwalkers:
 
Sequentielles Lesen eines Bitfeldes und Konvertieren von Teilen in eine natürliche Zahl,
bzw. invers Schreiben einer Ganzzahl in ein Bitfeld.
 
Darstellung der Aufgabe
 
Gegeben ist ein bitorientiertes Feld im Speicher.
Bei einer byteorientierten Darstellung im Speicher hätte das höchstwertige Bit
in Big-Endian Notation den Index 0.
Der Bit-Index ist streng monoton steigend mit einer Distanz von 1 zwischen den Indices.
Gelesen und geschrieben werden jeweils natürliche Zahlen in Big-Endian Notation.
Der Werteberreich der natürlichen Zahlen wird durch die zur Verfügung gestellte Bitbreite
im Bitfeld begrenzt.
 
Es sollen zwei Zugriffsmechanismen zur Verfügung gestellt werden:
Wahlfreier lesender und schreibender Zugriff auf das Bitfeld
Sequentielles Lesen/Schreiben im Bitfeld startend bei einem definierten Index.
Jeder Lese-/Schreibvorgang ``verbraucht'' die gerade verwendeten Bits.
 
Die Funktionen sollen in plain-C reentrant realisiert werden.
 
Kontext des Bitwalkers:
 
In ETCS werden Big-Endian notierte Telegramme zur Kommunikation zwischen Fahrzeug und
Stecke verwendet.
Im Rahmen von openETCS wird ein Generator entwickelt,
der die Telegrammspezifikation automatisiert in einen
Encoder/Decoder umsetzt und hierfür den Bitwalker als Hilfsfunktion nutzt.
\end{quote}

\selectlanguage{american}

\fxfatal{describe value of this description}

\fxfatal{what is missing in this description?}

\fxfatal{what does \emph{reentrant} exactly mean?}

\fxfatal{what does \emph{consumption of bit} mean?}

The \bitwalker software can be considered to consist of a \emph{public}
and a \emph{private} part.
\begin{itemize}
\item
The public part consists of the \isoc data type \bitwalkertype and several \isoc functions
that can be used to manipulate objects this type.

\item
The private part consists of the functions \peek and \poke
that implemnet the core functionality of \bitwalker.
\end{itemize}

Before we informally specify \bitwalker
we introduce some auxiliary concepts and formulate general assumptions.
We would also like to point out the following:
When we speak in this document of \emph{integers},
then we refer to the infinite set of mathematical
integers $\{\ldots, -1, 0, 1, \ldots\}$
and not to one of the many finite representation provided by the type system of \isoc.

This distinction is important because it usually makes more sense
to describe the functionality of a piece of software in a more
asbtract way.
In a later step the realities of specific the \isoc type system
have to be taken into account.

