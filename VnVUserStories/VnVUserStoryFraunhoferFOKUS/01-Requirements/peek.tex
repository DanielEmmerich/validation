\subsection{The Function \texttt{Bitwalker\_Peek}}

The function \texttt{Bitwalker\_Peek} reads a bit sequence from a bit stream
and converts it to an integer.

Its function signature reads as follows:

\begin{lstlisting}[style=acsl-block]
uint64_t  Bitwalker_Peek(unsigned int Startposition, 
                         unsigned int Length,
                         uint8_t Bitstream[],
                         unsigned int BitstreamSizeInBytes);
\end{lstlisting}


\subsubsection*{Arguments}

\begin{itemize}
    \item \texttt{Startposition} is the bit index in the bit stream 
    where the bit sequence starts.
    \item \texttt{Length} is the length of the bit sequence.
    \item \texttt{Bitstream} is the array which provides the bit stream.
    \item \texttt{BitstreamSizeInBytes} is the length of the array 
    containing the bit stream. 
\end{itemize}

\subsubsection*{Preconditions}

The following preconditions shall hold for the function arguments:

\begin{itemize}
\item \texttt{Bitstream} is a valid array of length \verb"BitstreamSizeInBytes"

\item \texttt{Length} $\leq$ \texttt{64} and

\item \texttt{Startposition + Length} $\leq$ \verb"UINT_MAX".
\end{itemize}

Note that additional constraints are implicitly expressed by the use
of \emph{unsigned} integer types.


\subsubsection*{Description}

The function \texttt{Bitwalker\_Peek} reads a bit sequence from a bit stream
and converts it to a 64-bit unsigned integer.

The left most bit of the bit sequence is interpreted as
the most significant bit.
Thus, for a bit sequence $(b_0, b_1,\ldots,b_{n - 1})$ the function
returns the sum
\begin{align}
    b_0 \cdot 2^{n - 1} + b_1\cdot 2^{n - 2} + \ldots + b_{n-1}\cdot 2^0
    =
    \sum_{i=0}^{n-1} b_i \cdot 2^{(n - 1) - i} 
\end{align}

If the bit sequence is not valid, then the function returns \texttt{0}.
This increases the robustness of the function.

