

\chapter{The public interface of \bitwalker}

\section{The structure \bitwalkertype}


\begin{lstlisting}[style=acsl-block]
   typedef struct
   {
       uint8_t       *Bitstream;
       unsigned int  Length;
       unsigned int  CurrentBitposition;
   } T_Bitwalker_Incremental_Locals;
\end{lstlisting}

\fxfatal{briefly describe struct members}

\clearpage

\section{Informal Specification of \inl{Bitwalker_IncrementalWalker_Init}}

In this section we specify the function \init.  The function  initializes object of the type \locals.
The function signature reads:


\begin{lstlisting}[style=acsl-block]
void Bitwalker_IncrementalWalker_Init(
        T_Bitwalker_Incremental_Locals *Locals,
        uint8_t Bitstream[],
        unsigned int Size,
        unsigned int FirstBitposition);
\end{lstlisting}


\paragraph{Arguments}
\begin{itemize}
    \item  \inl{Locals} is a dereferenceable pointer.
   \item \inl{Bitstream} is  the start address of a valid bit stream.
   \item \inl{Size} is the length of the bit stream, that starts at \inl{Bitstream}.
   \item \inl{FirstBitposition} is a valid bit index in the bit stream given by \inl{Bitstream} an d \inl{Size}
\end{itemize}

\paragraph{Preconditions}
\begin{itemize}
    \item  \inl{Locals} must me dereferenceable.
\end{itemize}

\paragraph{Description}\hfill\\\hfill
The function \init assigns:
\begin{itemize}
    \item \inl{Bitstream}  to \inl{Locals->Bitstream}.
    \item \inl{Size} to \inl{Locals->Length}
    \item \inl{FirstBitposition} to \inl{Locals->CurrentBitposition}
\end{itemize}
\fxfatal{\inl{Size} is length of the \inl{Bitstream} not of a \inl{sequence}, So \inl{Locals->Length} should be named \inl{Locals.Size}}

\clearpage

\section{Informal Specification of \inl{Bitwalker_IncrementalWalker_Peek_Next}}

\begin{lstlisting}[style=acsl-block]
uint64_t Bitwalker_IncrementalWalker_Peek_Next (T_Bitwalker_Incremental_Locals *Locals,
                                                unsigned int Length);
\end{lstlisting}


\paragraph{Arguments}
\begin{itemize}
    \item  \inl{Locals} holds a bit stream, the start position and the length of the bit stream.
   \item \inl{Length} is the Length of the sequence.
\end{itemize}

\paragraph{Preconditions}
\begin{itemize}
    \item  \inl{Locals} is a valid record, where \inl{Local.Bitstream} is a valid array of \inl{Locals.Length}.
    \item \inl{Length} $\leq$ \inl{64} and
    \item \inl{Locals.CurrentBitposition} $\leq $ \inl{UINT_MAX} $-$ \inl{Length}
\end{itemize}

\paragraph{Description}

\peeknext reads a sequence from a bit stream and  returns it as $64$ bit integer. If the bit sequence is not valid the function shall return \inl{0}. Furthermore  it sets \inl{Locals.CurrentBitposition} to the first position behind the sequence that was read. All other components of \inl{Locals} are not altered.




\clearpage

\section{Informal Specification of \inl{Bitwalker_IncrementalWalker_Peek_Finish}}

\begin{lstlisting}[style=acsl-block]
int Bitwalker_IncrementalWalker_Peek_Finish (T_Bitwalker_Incremental_Locals *Locals);
\end{lstlisting}

\paragraph{Arguments}

\begin{itemize}
    \item  \inl{Locals} holds a bit stream, the start position and the length of the bit stream.
\end{itemize}

\paragraph{Preconditions}
\begin{itemize}
    \item  \inl{Locals} is a valid record.
    \item \inl{Locals.CurrentBitposition} $\leq $ \inl{UINT_MAX}.
\end{itemize}

\paragraph{Description}

\peekfinish  returns the current bit position of the bit stream that is hold by \inl{Locals}.

\fxfatal{Should the return value be an unsigned type?}

\clearpage

\section{Informal Specification of \inl{Bitwalker_IncrementalWalker_Poke_Next}}
\begin{lstlisting}[style=acsl-block]
int		 Bitwalker_IncrementalWalker_Poke_Next(T_Bitwalker_Incremental_Locals *Locals, unsigned int Length, uint64_t Value);
\end{lstlisting}

\paragraph{Arguments}

\begin{itemize}
    \item  \inl{Locals} holds a bit stream, the start position of the sequence within the bit stream.
   \item \inl{Length} is the Length of the sequence.
   \item \inl{Value} is the integer which shall be converted into a bit sequence
\end{itemize}

\paragraph{Preconditions}
\begin{itemize}
    \item  \inl{Locals} is a valid record.
    \item \inl{Locals.CurrentBitposition} + \inl{Length} is less than \inl{UINT_MAX}
\end{itemize}

\paragraph{Description}

We specify \pokenext as follows:
The function \poke converts a 64-bit unsigned integer to a bit sequence and 
writes it into a bit stream.

For $0 \leq x$ exists a shortest sequence of~0 and~1
$(b_0, b_1,\ldots,b_{n - 1})$
such that
\begin{align}
    \sum_{i=0}^{n-1} b_i \cdot 2^{(n - 1) - i} = x.
\end{align}

The function \pokenext tries to store the sequence $(b_0, b_1,\ldots,b_{n - 1})$
in the bit sequence of \inl{Length} bits that starts
at bit index \inl{Locals.CurrentBitposition}.

The return value depends on  the following cases:
\begin{itemize}
    \item  If the bit sequence is not valid \peeknext  returns $-1$.
    \item  If the bit sequence is valid, then there are two cases:
\begin{itemize}
\item
If $x$ is greater or equal than $2^\mathtt{Length}$, then~$x$
cannot be represented as bit sequence $(b_0, b_1,\ldots,b_{\mathtt{Length} - 1})$.
\pokenext returns then~$-2$.

\item
If $x$ is less the $2^{\mathtt{Length}}$, then  the sequence
$(\overbrace{0,\ldots,0}^{\mathtt{Length}-n},b_0, b_1,\ldots,b_{n - 1})$
is stored in the bit stream starting at \inl{Locals.CurrentBitposition}.
The return value of \pokenext is 0.

\end{itemize}
\end{itemize}

Regardless of whether the poke was successful \pokenext sets the value  \inl{Locals.CurrentBitposition} to the first position behind the sequence that it tired to poke.
 All other components of the record \inl{Locals} remain unaltered.


\clearpage

\section{Informal Specification of \inl{Bitwalker_IncrementalWalker_Poke_Finish}}

\begin{lstlisting}[style=acsl-block]
int		 Bitwalker_IncrementalWalker_Poke_Finish (T_Bitwalker_Incremental_Locals *Locals);
\end{lstlisting}

\paragraph{Arguments}
\begin{itemize}
    \item  \inl{Locals} holds a bit stream, the start position and the length of the bit stream.
\end{itemize}

\paragraph{Preconditions}
\begin{itemize}
    \item  \inl{Locals} is a valid record.
    \item \inl{Locals.CurrentBitposition} $\leq $ \inl{UINT_MAX}.
\end{itemize}

\paragraph{Description}

\peekfinish  returns the current bit position of the bit stream that is hold by \inl{Locals}.


