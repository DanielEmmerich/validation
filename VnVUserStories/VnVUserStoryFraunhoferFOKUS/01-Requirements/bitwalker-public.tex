
\section{The public interface of \bitwalker}
\label{sec:bitwalker-public}

This section describes the public interface of \bitwalker.

\subsection{The structure \bitwalkertype}


The type \bitwalkertype is defined as follows\\[1em]

\begin{lstlisting}[style=acsl-block]
   struct  T_Bitwalker_Incremental_Locals
   {
       uint8_t       *Bitstream;
       unsigned int  Length;
       unsigned int  CurrentBitposition;
   };
\end{lstlisting}


\paragraph{Description}~

\begin{itemize}

   \item \inl{Bitstream} is  the start address of a valid bit stream.
   \item \inl{Length} is the length of the bit stream, that starts at \inl{Bitstream}.
   \item \inl{CurrentBitposition} is a valid bit index in
              the bit stream given by \inl{Bitstream} and \inl{Length}

\end{itemize}

\paragraph*{Remark}
The field \inl{Length} will be passed to \peek and \poke as \inl{BitstreamSize}.
This might be confusing because those functions also have an argument 
named~\inl{Length}.

\clearpage

\subsection{\inl{Bitwalker_IncrementalWalker_Init}}

In this section we specify the function \init.
 The function  initializes object of the type \bitwalkertype.
The function signature reads:\\[1em]


\begin{lstlisting}[style=acsl-block]
void Bitwalker_IncrementalWalker_Init(
        T_Bitwalker_Incremental_Locals *Locals,
        uint8_t Bitstream[],
        unsigned int Size,
        unsigned int FirstBitposition);
\end{lstlisting}


\paragraph{Preconditions}
\begin{itemize}
   \item  \inl{Locals} is a dereferenceable pointer.
   \item \inl{Bitstream} is  the start address of a valid bit stream.
   \item \inl{Size} is the length of the bit stream, that starts at \inl{Bitstream}.
   \item \inl{FirstBitposition} is a valid bit index in the bit stream given by \inl{Bitstream} and \inl{Size}
\end{itemize}

\paragraph{Description}~

The function \init assigns
\begin{itemize}
    \item \inl{Bitstream}  to \inl{Locals->Bitstream}.
    \item \inl{Size} to \inl{Locals->Length}
    \item \inl{FirstBitposition} to \inl{Locals->CurrentBitposition}
\end{itemize}

\clearpage

\subsection{\inl{Bitwalker_IncrementalWalker_Peek_Next}}

The function \peeknext reads a sequence from a bit stream 
and increments the current position in the bit stream by the 
length of the read bit sequence.
Its function signature reads as follows:\\[1em]


\begin{lstlisting}[style=acsl-block]
uint64_t Bitwalker_IncrementalWalker_Peek_Next(
    T_Bitwalker_Incremental_Locals *Locals,
    unsigned int Length);
\end{lstlisting}


\paragraph{Preconditions}
\begin{itemize}
  \item  \inl{Locals} must be dereferenceable
  \item \inl{Length} is the length of the bit sequence and shall be less or equal~64
  \item \inl{Locals->CurrentBitposition} $\leq$ \inl{UINT_MAX} $-$ \inl{Length}
\end{itemize}

\paragraph{Description}~

\peeknext reads a bit sequence from a bit stream and returns it as~64 bit integer.

\fxfatal{describe more precisely, see \peek}

If the bit sequence is not valid the function shall return \inl{0}.

The function increments the value \inl{Locals->CurrentBitposition} by \inl{Length}.

\fxfatal{does it make sense to increment in the case of an invalid bit sequence?}

\clearpage

\subsection{\inl{Bitwalker_IncrementalWalker_Peek_Finish}}


 The function signature reads:\\[1em]

\begin{lstlisting}[style=acsl-block]
int Bitwalker_IncrementalWalker_Peek_Finish(
        T_Bitwalker_Incremental_Locals *Locals);
\end{lstlisting}

\paragraph{Preconditions}
\begin{itemize}
   \item  \inl{Locals} must be dereferenceable
\end{itemize}

\paragraph{Description}~

The function  returns \inl{Locals->CurrentBitposition}.
It does not change any variables.

\paragraph*{Remark} The return value of this function is \inl{int} despite
the fact that it returns a value of type \inl{unsigned int}.
This does not make much sense and should best be avoided.

\clearpage

\subsection{\inl{Bitwalker_IncrementalWalker_Poke_Next}}

The function \pokenext writes a bit sequence into a bit stream 
and increments the current position in the bit stream by the 
length of the read bit sequence.\\[1em]


\begin{lstlisting}[style=acsl-block]
int Bitwalker_IncrementalWalker_Poke_Next(
        T_Bitwalker_Incremental_Locals *Locals,
        unsigned int Length,
        uint64_t Value);
\end{lstlisting}


\paragraph{Preconditions}
\begin{itemize}
    \item  \inl{Locals} must be dereferenceable
    \item \inl{Locals->CurrentBitposition} + \inl{Length} is less than or equal \inl{UINT_MAX}
\end{itemize}

\paragraph{Description}~

We specify \pokenext as follows:
The function \poke converts a 64-bit unsigned integer to a bit sequence and 
writes it into a bit stream.

For $0 \leq x$ exists a shortest sequence of~0 and~1
$(b_0, b_1,\ldots,b_{n - 1})$
such that
\begin{align}
    \sum_{i=0}^{n-1} b_i \cdot 2^{(n - 1) - i} = x.
\end{align}

The function \pokenext tries to store the sequence $(b_0, b_1,\ldots,b_{n - 1})$
in the bit sequence of \inl{Length} bits that starts
at bit index \inl{Locals->CurrentBitposition}.

The return value depends on  the following cases:
\begin{itemize}
    \item  If the bit sequence is not valid \peeknext  returns $-1$.
    \item  If the bit sequence is valid, then there are two cases:
\begin{itemize}
\item
If $x$ is greater or equal than $2^\mathtt{Length}$, then~$x$
cannot be represented as bit sequence $(b_0, b_1,\ldots,b_{\mathtt{Length} - 1})$.
\pokenext returns then~$-2$.

\item
If $x$ is less the $2^{\mathtt{Length}}$, then  the sequence
$(\overbrace{0,\ldots,0}^{\mathtt{Length}-n},b_0, b_1,\ldots,b_{n - 1})$
is stored in the bit stream starting at \inl{Locals->CurrentBitposition}.
The return value of the function \pokenext is 0.

\end{itemize}
\end{itemize}

Regardless of whether the poke was successful \pokenext sets the value  \inl{Locals->CurrentBitposition} to the first position behind the sequence that it tired to poke.
 All other components of the record \inl{Locals} remain unaltered.


\fxfatal{does it make sense to increment in the case of an invalid bit sequence?}

\clearpage

\subsection{\inl{Bitwalker_IncrementalWalker_Poke_Finish}}

 The function signature reads:\\[1em]

\begin{lstlisting}[style=acsl-block]
int Bitwalker_IncrementalWalker_Poke_Finish(
        T_Bitwalker_Incremental_Locals *Locals);
\end{lstlisting}

\paragraph{Preconditions}
\begin{itemize}
    \item  \inl{Locals} must be dereferenceable
\end{itemize}

\paragraph{Description}~

The function  returns \inl{Locals->CurrentBitposition}.

\paragraph{Remark*}
The functions \peekfinish and\\
\pokefinish have the same specification.
Moreover, both function return an object of type \inl{unsigned int} as \inl{int}
for which no convincing reason is available.





