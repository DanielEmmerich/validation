\documentclass[11pt, a4paper]{article}

\usepackage[utf8]{inputenc}
\usepackage{graphicx}
\usepackage{color}
\usepackage{todo}
\usepackage{booktabs}
\usepackage{geometry}
\usepackage[hidelinks]{hyperref}
\graphicspath{{figures/pdf/}}


\newgeometry{margin=3cm}
\title{V\&V Users Stories : Bremen University, DLR, Siemens
\\
\normalsize{System Integration Testing with Model Based Testing}
}
\author{Cécile Braunstein \and Jan Peleska}
\date{\today}

% Makes Marginpars easier to read
\setlength{\marginparwidth}{1in}

\pagestyle{myheadings}
\markright{openETCS \hfill {\tiny This work is licensed under a Creative Commons Attribution-ShareAlike 3.0 Unported License} \hfill}

\begin{document}

\maketitle

\begin{abstract}
This document presents the experiments (to be) conducted by Uni Bremen, DLR
and Siemens  for the system integration testing of parts of the OBU. 
The method details may be found in the WP4 Verification and Validation
Plan. This document summarize the task to achieve  and the state of
this task.
\end{abstract}



\section{Document and folder description}
\paragraph{01-Requirements}
\begin{itemize}
  \item WP4-Validation and Verification plan : Contains the methods
  definition and the details of our task
  \item SysML Management of Radio Communication model : SysML model of
    the MoRC in xml format generated from an Enterprise architect description
  \item SysML Management of Radio Communication doc : Explanation of
    the SysML model, contains also the interface definition of the
    MoRC. 
\end{itemize}

\section{Task Resume}
The detail of the task may be found in the Verification and Validation
plan.

Uni Bremen will provide two SysML test model that represent the
Management of the radio communication and the ceiling speed monitoring
of the OBU. These models will be used to automatically generate test
cases with the RT-tester tool. The tests will be used for system
integration testing. Our goal is to have an test environment within
the DLR laboratory for Hardware in the loop test and also use the same
generated test cases to perform software int the loop testing with the
generated code from SCADE.
To show the strength of our tests, the test suites automatically
generated will be analyzed according to the well known coverage
measure. Moreover we will compare our test to those defined in the SUBSET-76.

\section{Status of the task}
\subsection{Providing Test environment}
\begin{tabular}{p{.5\textwidth}cp{.3\textwidth}}\toprule
Activity & Status & Remarks \\\midrule
 Create a test model in SysML MoRC & Done &  From the model evaluation
 of WP7\\\midrule
Provide the test model for the ceiling speed monitoring & To be
commited & Still some cleaning to do \\\midrule
Generate test cases according to the defined interface given by
  DLR. (RT-Tester) & Active & DLR confirm interface \\ \midrule
Provide simulation environment for the
  track-to-train simulation  along routes used
  for testing & Active & \\\midrule
Study the automatic generation of these track layouts and speed
  simulations & & \\\midrule
Set up a test environments for
Hardware-in-the-loop Testing within DLR laboratory & & \\\midrule
Set up a test environments for Software-in-loop testing with code
provided by SCADE (Siemens) & & \\
\bottomrule
\end{tabular}

\subsection{Test cases generation}
Together with generated the test cases we will study the quality of
the test suites.

\begin{tabular}{p{.5\textwidth}cp{.3\textwidth}}\toprule
Activity & Status & Remarks \\\midrule
Check the test model (SysML) RTT-BMC  &&\\\midrule
Add relevant LTL properties & Active&\\\midrule
Structural coverage analysis & Done & Automatically performs by  RT-Tester\\\midrule
 Requirement coverage analysis &&\\\midrule
 Mutation coverage  analysis &&\\\midrule
 Provide techniques and Howto describing how test cases from
  Subset 76 can be executed in the RT-Tester environment &&\\\midrule
 Compare new test cases created by RT-Tester to new test cases
  for ceiling speed monitoring provided by ERTMS standardization
  group &&\\
\bottomrule
\end{tabular}

\subsection{Exchange Formats}
Test models represented in XMI/Ecore are used as SysML test modeling
standard. RT-Tester model parsers are extended to cope with this
format.

\begin{tabular}{p{.5\textwidth}cp{.3\textwidth}}\toprule
Activity & Status & Remarks \\\midrule
Test procedures general abstract syntax definition &Active&\\\midrule
Test results (test execution logs) general format &&\\
\bottomrule
\end{tabular}

\bibliographystyle{alpha}
\bibliography{biblio} 
\end{document}

%  LocalWords:  cp
