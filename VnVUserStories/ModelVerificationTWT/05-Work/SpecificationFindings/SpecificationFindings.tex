\documentclass{template/openetcs_article}
% Use the option "nocc" if the document is not licensed under Creative Commons
%\documentclass[nocc]{template/openetcs_article}
\usepackage{lipsum,url}
\usepackage{supertabular}
\usepackage{multirow}
\usepackage{color, colortbl}
\definecolor{gray}{rgb}{0.8,0.8,0.8}
\graphicspath{{./template/}{.}{./images/}}

\begin{document}

\frontmatter
\project{openETCS}

%Please do not change anything above this line
%============================


% The document metadata is defined below

%assign a report number here
\reportnum{OETCS/WP1/D02}

%define your workpackage here
\wp{Work-Package 4: ``V\&V''}

%set a title here
\title{ETCS Specification Findings}

%set a subtitle here
\subtitle{Findings of ETCS specification analyses}

%set the date of the report here
\date{\today}

%define a list of authors and their affiliation here

\author{Stefan Rieger}

\affiliation{TWT GmbH Science \& Innovation\\
  Ernsthaldenstraße 17\\
  70565 Stuttgart\\
  Germany}

%\author{Fabien~Belmonte}

%\affiliation{Alstom Transport\\
%  Information Solution\\
%  48 rue Albert Dhalenne\\
%  93482 Saint-Ouen cedex, France}

% define the coverart
\coverart[width=350pt]{openETCS_EUPL}

%define the type of report
\reporttype{Description of work}


\begin{abstract}
  This document lists analysis results of the ETCS specification and accompagnying standards that indicate problems such as unclearities, inconsistencies, ambiguities, incompleteness or errors. For now it is part of TWT's model verification user story but the goal is to extend its scope.
\end{abstract}

\newcounter{issuecounter}
\newcommand{\issue}[1]{\refstepcounter{issuecounter}\textbf{Issue \#\arabic{issuecounter} (#1):}}	

\newcommand{\resolution}{\textit{Resolution: }}	

%=============================
\maketitle

\tableofcontents
\newpage
%=============================



\section{Purpose of this Document}

  This document lists findings in the ETCS specification and accompagnying standards indicating problems such as inconsistencies, ambiguities, incompleteness or errors that arise during analysis or modelling. The goals are the following:
  \begin{itemize}
    \item Clarify and correct problems to help in system modelling
    \item Indicate issues in the standard for future improvement
    \item ... 
  \end{itemize}

This document is to be considered as ``living document'' that is continuously extended during the runtime of the project. Solutions to issues or workarounds shall be added when available.



\section{List of Issues}

\subsection{Subset 026 3.6 Location Principles, Train Position and Train Orientation}

\issue{3.6.1.3 Train Position}
What is the difference between the \emph{estimated train front end position} and the \emph{train confidence interval}? Both values are contained in the \emph{train position information}. It seems that the \emph{train confidence interval} is a more conservative approximation. If this is the case, how exactly is the \emph{estimated train front end position} defined?

\resolution Write resolution here...

\subsection{Subset 026 5.x}

\issue{Semantics of the tables specifying the update of on-board variables}
Do I read each row of such a table as ``If transition condition holds and a variable a the value as shown in the table, then change the value of that variable according to the table'' (i.e., the variable does not necessarily have the value as shown in the table) or ``If transition condition holds, then change the value of a variable according to the table'' (i.e., each variable has the value as shown in the table).

\resolution The semantics is unclear and the values should be checked.


\subsection{Subset 026 5.4 Procedure Start of Mission}

\issue{5.4.2.2 Train Data}
The specification seems to be inconsistent:
In addition to the data listed, the control flow also depends on the status of the virtual balise cover and the radio network ID.

Our guess: radio network ID = RBC-ID

%\issue{5.4.3.2 State S0 - Driver starts a mission} We assume that this also refers to a signal sent by the driver to the on-board unit (e.g., by pushing a button).

\issue{5.4.3.2 State S0 - communication session}
Can there be an active communication session other than with the RIU and the RBC?

\issue{5.4.3.2 State S1 - Driver-ID Validation}
The specification states that the driver revalidates the Driver-ID. So it can be assumed that the system relies on correct validation by the driver. Is this true? 
In other words, what happens if the driver enters an invalid Driver-ID?

\resolution It is true. There are up to three iterations of asking the driver for a correct ID.

\issue{5.4.3.2 State S1 - Driver-ID Validation}
 When does the Driver-ID become invalid or unknown? At End of Mission? The same holds for the \emph{train running number}.


\issue{5.4.3.2 State S1 - Virtual Balise Cover}
Virtual balise cover is not properly specified. The same holds for the process of setting/removing virtual balise cover. How does setting and removing a virtual balise cover change the status of the virtual balise cover? Our guess is that setting results in a valid virtual balise cover whereas removing changes the status to unknown.

\issue{5.4.3.2 State S1 - Transition E1}
Last paragraph, S1: How do I understand ``possibly further to Train running number ...''?

\issue{5.4.3.2 State S2 - Missing case: level data is valid}
There seems to be an inconsistency: It is not explicitly mentioned, that the paragraphs 3 \& 4 are related to the 'valid' case. This case is possible, see D2 but it is not mentioned how to proceed. We assume that also in this case para 3 and 4 in S2 are applied.



\issue{5.4.3.2 State S2 - Enter/Re-validate Level}
The specification distinguishes the following three cases:
\begin{enumerate}
   \item Entering level (if state \emph{unknown})
   \item Re-validate level (if state \emph{invalid})
   \item Re-enter level (if state \emph{invalid})
\end{enumerate}
The purpose of this distinction is not clear as entering the level suffices (the current setting is invalid or unknown and thus irrelevant).

\resolution It depends on the actual interface of the driver whether we need to distinguish between validation and reentering.

\issue{5.4.3.2 State S3 - Selection of Radio Network}
Inconsistency in the specification: It is not mentioned that a valid \emph{radio network ID} must be stored in the on-board unit, but this seems to be necessary because the driver may not select a new radio network. The type of the radio network IDs is not mentioned.

\issue{5.4.3.2 State S3 - Mobile terminal}
What happens if no Mobile terminal is registered to a Radio Network (or can we assume that at least one mobile terminal is always registered); see 2nd para in S3.

\issue{5.4.3.2 State S3 - RBC-ID}
What is the difference between the \emph{RBC-ID} that may be invalid and the \emph{last stored RBC-ID}? Why can the latter not be invalid?



\issue{5.4.3.2 State S3}
In item 2 in S3, the driver sets a radio network ID and RBC-ID to unknown but in item 3, it is assumed that the RBC-ID is invalid. So is it possible to skip the first two items? In addition, after applying item 3 (i.e., using the last stored RBC-ID number) is the RBC-ID set to valid?

\issue{5.4.3.2 State S3 - EIRENE Short Number}
The option to use the EIRENE short number is unclear: I assume the RBC is triggered to send an RBC-ID (i.e., the respective variable is set to valid). However, what happens if something goes wrong? We read in 3.18.4.3.4.1:  ``... does not direct to a RBC with the stored RBC ID, the connection will be terminated''. So how does the flow continue and which state do we enter in this case?

\issue{5.4.3.2 State S3 - missing cases}
How do we proceed if the driver decides not to reenter the radio network ID? Likewise, how do we proceed if no mobile terminal is registered?

\issue{5.4.3.2 State D7 - Mobile Terminal registered}
The definition of a mobile terminal is missing. How many mobile terminals are there? Is this equivalent to radio network registration (see state S3)? The latter is assumed for the initial models.

\issue{5.4.3.2 State D33}
Does the RBC send a signal showing whether it can validate the position report (see para 1 and 3 in D33)?

\issue{5.4.3.2 State S10}\label{i:s10}
The point in time when item 4 takes place is unclear for E12 (i.e., item 1): I assume the driver selects SH (i.e., the mode is changed), then item 4 is applied (invalid position is set to unknown), then the level check (see item 1) is performed, and based on this check procedure shunting is called or the process goes back to S10. 

Moreover, how do I read the last sentence in item 1: Does the RBC reject the request for shunting if the level is 2 or 3 or can the RBC always reject the request for shunting but there will be a special treatment in case the level is 2 or 3? 

The same question arises at 5.4.5.3.g.

\issue{5.4.3.2 State S11 - RBC acknowledgment}
Can we always expect an acknowledgment? (Sometimes it is specified that the RBC is, after some timeout, triggered again.) 
\marginpar{Stephan}

\issue{5.4.3.3, last row on p.17}
I assume that if the driver chooses to re-enter the level (see 5.4.5.3.e), then the values are set according to the table, and after he entered the level, at least the status of the ETCS level should be set to valid.

\issue{5.4.5.1}
What is meant by ``above D11'' in a flowchart where flow also goes from left to right, and what is meant by ``the nominal procedure applies''?

\issue{5.4.5.2}
Where is the missing information specified?

\issue{5.4.5.3 a}
Does ``if the position is still invalid'' (line 3) refer to a condition that is checked after the procedure Override has been executed?

\issue{5.4.5.3 b,c,i}
How does the process continue---as described in state S1?

\issue{5.4.5.3 f}
How does the procedure continue; that is, to which state do we go? (Assumption: The mission is considered as started, see 5.4.6.1.) 


\subsection{Subset 026 5.5 Procedure End of Mission}

\issue{5.5.3.1.3 - report to RBC}
Is End of Mission reported to RBC in every level or only in ETCS level 2 and 3. If yes, what happens in the situation described in 5.5.4.1.1?

\issue{5.5.4.1.2}
By 5.5.3.1.2 and 5.5.3.1.3, this should hold for every ETCS level and not only for 2 and 3. 



\subsection{Subset 026 5.6 Procedure Shunting Initiated by Driver}

\issue{5.6.2.2 A030 - calling Trip procedure}
We call the train trip procedure which itself calls Shunting. How do we ensure that this recursion eventually stops?

\issue{5.6.2.2 D040 - ongoing mission}
What is an ongoing mission?

\issue{5.6.4.1.2 -- termination}
What happens after the driver has been informed; that is, do we continue as we would do in case the session was not terminated (i.e., as in A220)? The same question arises for the situation described in 5.6.4.3.

\issue{5.6.4.1.3}
Does this item refer to a transition directly after A220 or to 5.6.4.1.2?


%\subsection{Subset 026 5.8 Procedure Override}



\subsection{Subset 026 5.11 Procedure Train Trip}

\issue{5.11.2.2 A025}
Considering the flow chart and D020, I assume the process shall go to D020 rather that A030.

\issue{5.11.2.2 D80}
Replace with D080.

\issue{5.11.2.2 S130}
After having acknowledged mode change to PT (see S120), the RBC is assumed to revoke all pending emergency stops. Is the RBC triggered by the on-board unit to do so (e.g., after D130) or does S120 serve as the trigger?

\issue{5.11.4.1.2 - termination}
How do we proceed after the communication session has been terminated?

\issue{5.11.4.2 - override}
How do we proceed after override?

\subsection{Procedure On Sight and Comparable procedures}

\issue{Delayed messages}
What happens if the acknowledgment of the driver is received after the train is already braking due to overpassing the safe front?

\subsection{Interplay of the procedures}

\issue{How to restart a mission?} 
Probably a new Mission can be started after the ``End of Mission'' procedure.
To enter the ``Start of Mission'' procedure, the mode must be Standby (SB).
This is not necessarily the case after finishing End of Mission. 
Can the Train enter SB-mode after End of Mission and if so, how?

\issue{State after ``Train Trip'' \& ``Override''}
Which procedures can be called after having finished the Override/Train Trip procedure?


\bibliographystyle{unsrt}
\bibliography{erdc}


%\begin{thebibliography}{9}

%\bibitem{lamport94}
%  Leslie Lamport,
%  \emph{\LaTeX: A Document Preparation System}.
%  Addison Wesley, Massachusetts,
%  2nd Edition,
%  1994.

%\end{thebibliography}

%===================================================
%Do NOT change anything below this line

\end{document}
