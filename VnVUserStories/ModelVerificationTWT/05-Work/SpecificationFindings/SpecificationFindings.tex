\documentclass{template/openetcs_article}
% Use the option "nocc" if the document is not licensed under Creative Commons
%\documentclass[nocc]{template/openetcs_article}
\usepackage{lipsum,url}
\usepackage{supertabular}
\usepackage{multirow}
\usepackage{color, colortbl}
\definecolor{gray}{rgb}{0.8,0.8,0.8}
\graphicspath{{./template/}{.}{./images/}}
\begin{document}
\frontmatter
\project{openETCS}

%Please do not change anything above this line
%============================



% The document metadata is defined below

%assign a report number here
\reportnum{OETCS/WP1/D02}

%define your workpackage here
\wp{Work-Package 4: ``V\&V''}

%set a title here
\title{ETCS Specification Findings}

%set a subtitle here
\subtitle{Findings of ETCS specification analyses}

%set the date of the report here
\date{September 2013}

%define a list of authors and their affiliation here

\author{Stefan Rieger}

\affiliation{TWT GmbH Science \& Innovation\\
  Bernh\"auser Stra\"se 40-42\\
  73765 Neuhausen\\
  Germany}

%\author{Fabien~Belmonte}

%\affiliation{Alstom Transport\\
%  Information Solution\\
%  48 rue Albert Dhalenne\\
%  93482 Saint-Ouen cedex, France}

% define the coverart
\coverart[width=350pt]{openETCS_EUPL}

%define the type of report
\reporttype{Description of work}


\begin{abstract}
  This document lists analysis results of the ETCS specification and accompagnying standards that indicate problems such as unclearities, inconsistencies, ambiguities, incompleteness or errors. For now it is part of TWT's model verification user story but the goal is to extend its scope.
\end{abstract}

\newcounter{issuecounter}
\newcommand{\issue}[1]{\refstepcounter{issuecounter}\textbf{Issue \#\arabic{issuecounter} (#1):}}	

\newcommand{\resolution}{\textit{Resolution: }}	

%=============================
\maketitle

\tableofcontents
\newpage
%=============================



\section{Purpose of this Document}

  This document lists findings in the ETCS specification and accompagnying standards indicating problems such as inconsistencies, ambiguities, incompleteness or errors that arise during analysis or modelling. The goals are the following:
  \begin{itemize}
    \item Clarify and correct problems to help in system modelling
    \item Indicate issues in the standard for future improvement
    \item ... 
  \end{itemize}

This document is to be considered as ``living document'' that is continuously extended during the runtime of the project. Solutions to issues or workarounds shall be added when available.

\section{List of Issues}

\subsection{Subset 026 3.6 Location Principles, Train Position and Train Orientation}

\issue{3.6.1.3 Train Position}
What is the difference between the \emph{estimated train front end position} and the \emph{train confidence interval}? Both values are contained in the \emph{train position information}. It seems that the \emph{train confidence interval} is a more conservative approximation. If this is the case, how exactly is the \emph{estimated train front end position} defined?

\resolution Write resolution here...

\subsection{Subset 026 5.4 Procedure Start of Mission}

\issue{5.4.2.2 Train Data}
It is not clear what the train data consists of.

\issue{5.4.3.2 State S1 - Driver-ID Validation}
The specification states that the driver revalidates the Driver-ID. So it can be assumed that the system relies on correct validation by the driver. Is this true? When does the Driver-ID become invalid or unknown? At End of Mission? The same holds for the \emph{train running number}.

\issue{5.4.3.2 State S1 - Virtual Balise Cover}
Virtual balise cover is not properly specified. The same holds for the process of setting/removing virtual balise cover. For now this feature is omitted in the models.

\issue{5.4.3.2 State S2 - Enter/Re-validate Level}
The specification distinguishes the following three cases
\begin{enumerate}
   \item Entering level (if state \emph{unknown})
   \item Re-validate level (if state \emph{invalid})
   \item Re-enter level (if state \emph{invalid})
\end{enumerate}
The purpose of this distinction is not clear as entering the level suffices (the current setting is invalid or unknown and thus irrelevant).




\bibliographystyle{unsrt}
\bibliography{erdc}


\begin{thebibliography}{9}

%\bibitem{lamport94}
%  Leslie Lamport,
%  \emph{\LaTeX: A Document Preparation System}.
%  Addison Wesley, Massachusetts,
%  2nd Edition,
%  1994.

\end{thebibliography}

%===================================================
%Do NOT change anything below this line

\end{document}
