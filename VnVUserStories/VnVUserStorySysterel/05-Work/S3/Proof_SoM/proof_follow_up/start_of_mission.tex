\section{Proof of : Start Of Mission}
\subsection{Files used for the proof}
\begin{tabular}{|c|c|}
\hline
Used node & Property file \\ \hline
Procedures::Procedure\_StartOfMission & start\_of\_mission.hll \\
%\hline
%ManageModes & start\_of\_mission\_topnode.hll \\
\hline
\end{tabular}

\subsection{What we want to prove}
We want to prove that the procedure Procedure\_StartOfMission is a
correct implementation of the section ``5.6 Procedure Start of
mission'' (cf. D1).

The flowchart is not entirely modelized in the Scade model,
communication with the driver doesn't seem to be implemented.

Moreover, an implementation of emergency brake management is present in
the Scade model. This may be an implementation of ``Standstill
supervision''. Emergency brake can be triggered at any time during the
procedure execution and the train shall be at standstill to start the
procedure.

States modelized are \texttt{S0}, state \texttt{S10}, states
\texttt{S20} to \texttt{S25}, states \texttt{NL/SR/FS/OS/LS/SH/SN/UN
  mode} and state \texttt{See procedure ``Shunting initiated by
  Driver''}. 

A part of the actual flowchart is drawn by figure~\ref{som-flowchart},
only the part which is changed can be seen.

The ``On Going Mission'' variable, input of SH Initiated by Driver, is
forced to False. This is justified by SRS 5, section ``5.4.6 Entry to
Mode Considered as a Mission''.

\begin{figure}[h]
\centering
\begin{tikzpicture}[%
    >=triangle 60,              % Nice arrows; your taste may be different
    start chain=going below,    % General flow is top-to-bottom
    node distance=6mm and 60mm, % Global setup of box spacing
    edge from parent/.style={->,draw},
    ]
% ------------------------------------------------- 
% A few box styles 
% <on chain> *and* <on grid> reduce the need for manual relative
% positioning of nodes
\tikzset{
  base/.style={draw, on chain, on grid, align=center, minimum height=4ex},
  state/.style={base, rectangle, text width=8em},
  % -------------------------------------------------
  % Connector line styles for different parts of the diagram
  transition_label/.style ={base, rectangle, draw=none, thick, fill=none}
}

\node [state] (s0) {\texttt{S0}};
\node [state] (s10) {\texttt{S10}}
  child {node (c11) {...}}
  child {node (c12) {...}}
  child {node (c13) {...}};
%\node [transition_label] (tvalid) [right=of s10] {\texttt{Valid\_Train\_Data\_Stored}};
\node [state, right=of s10] (s20) {\texttt{S20}}
  child {node (c21) {...}}
  child {node (c22) {...}}
  child {node (c23) {...}};


\draw [->] (s0.south) -- (s10);
\draw [->] (s10.east) -- node [above] {\texttt{Valid\_Train\_}} node [below] {\texttt{Data\_Stored}} (s20);
%\draw [-] (s10.east) -- (tvalid);
%\draw [->] (tvalid.east) -- (s20);

\end{tikzpicture}
\caption{Part of actual flowchart of Start of Mission procedure.}
\label{som-flowchart}
\end{figure}

Also, assumptions are made for this procedure:
\begin{enumerate}
\item Level should not change : During Start of Mission procedure,
  Level should not change.

\item Train Data should not change : During Start of Mission
  procedure, train data should not be modified. Then, their validity
  should not change.
\end{enumerate}


\subsection{Results and conclusions}
Considering assumptions made, the Scade model of Procedure Start of
Mission does not correspond to the specification.

This is explained by the following differences between the
specification and the Scade model:
\begin{enumerate}
\item \label{s0-case} The state \texttt{S0} is not specified as a
  usual state. It is specified as a condition that trigger the
  procedure. In the Scade model, \texttt{S0} is implemented as a
  condition that must be \texttt{True} to engaged the Start of Mission
  procedure. When the condition is \texttt{False}, the procedure is
  stopped.

\item \label{sh-case} Same observations present in
  section~\ref{SH-results-and-conclusions}.
\end{enumerate}


Here is propositions to resolve previous problems:
\begin{description}
\item[Proposition for \ref{s0-case}.] The ambiguity of \texttt{S0} definition must be
  clarified : either it is a state with a clear action in its
  definition, or it is a condition which trigger or not the procedure.
\item[Proposition for \ref{sh-case}.] It is the same propositions that
  are made in section~\ref{SH-results-and-conclusions}.
\end{description}
