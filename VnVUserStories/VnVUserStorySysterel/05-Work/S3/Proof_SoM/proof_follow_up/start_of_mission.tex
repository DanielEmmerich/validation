\section{Proof of : Start Of Mission}
\subsection{Files used for the proof}
\begin{tabular}{|c|c|}
\hline
Used node & Property file \\ \hline
Procedures::Procedure\_StartOfMission & startofmission\_proof.hll \\
\hline
ManageModes & startofmission\_topnode\_proof.hll \\
\hline
\end{tabular}

\subsection{What is proved ?}
We want to prove that the procedure Procedure\_StartOfMission is a
correct implementation of the section 5.6 Procedure Start of mission
(cf. D1).

The flowchart is not entirely modelised in the Scade model and an
implementation of emergency brake management is present in the Scade
model.

States modelised are S0, from S10 and from S20. In fact, S0 is a
condition which could trigger S10 and S20 states or stop the procedure
execution at an time. 

Also, emergency brake can be triggered at any time during the procedure
execution.

The ``On Going Mission'' variable, input of SH Initiated by Driver, is
forced to False. This is justified by SRS 5, section ``5.4.6 Entry to
Mode Considered as a Mission''.

TODO : make the actual flowchart modelised in HLL

\paragraph{Constraints used}
\begin{enumerate}
\item Level should not change : During Start of Mission procedure,
  Level should not change.

\item Train Data should not change : During Start of Mission procedure,
  train data should not be modified. Then, their validity should not
  change.

\item The train shall be at standstill to start the pocedure.

\item Same constraint used for the procedure SH Initiated By Driver proof.
\end{enumerate}



\subsection{Results}
Considering constraints and our model, the Scade model of Procedure
Start of Mission corresponds to the specification.

%\subsection{Conclusion and questions}
