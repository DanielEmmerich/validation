\section{Proof of: Shunting Initiated By Driver}
\label{sh-initiated-by-driver}
\subsection{Files used for the proof}
\begin{tabular}{|c|c|}
\hline
Used node & Property file \\ \hline
Procedures::SH\_Initiated\_By\_Driver & shunting\_initiated\_by\_driver.hll \\
\hline
ManageModes & shunting\_initiated\_by\_driver\_topnode.hll \\
\hline
\end{tabular}


\subsection{What we want to prove}
We want to prove that the procedure SH\_Initiated\_By\_Driver is a
correct implementation of the section 5.6 Shunting Initiated By Driver
(cf. D1).

To prove this, a modelization of the flowchart is proposed in the
property file. However, the flowchart is not entirely modelized by the
Scade model, assumptions are taken:
\begin{enumerate} 
\item A030 and National Trip occurrence is covered by Trip conditions and
procedures with higher priority.

\item Communication with RBC is out of the scope of this function
  (repetition of message to RBC, waiting for acknowledgment, ...)
\end{enumerate} 
These assumptions have the following consequences: 
\begin{enumerate} 
\item \texttt{D030} and \texttt{A030} are not in this HLL model. This
  implies that we have to deal with the "NTC" transition leaving
  D020. To resolve this problem, we assume this transition is merged
  with the "0/1" transition.

\item \texttt{D080}, \texttt{A095}, \texttt{S100}, \texttt{A115} are
  not in this HLL model. This implies the "No" transition leaving D040
  is going nowhere. In this case, we assume this transition leads to
  an "End" state, which means that process shall end.
\end{enumerate} 

\subsection{Results and conclusions}
\label{SH-results-and-conclusions}
Considering assumptions, SH\_Initiated\_By\_Driver Scade model does
not correspond to the specification.

This is explained by the following differences between the
specification and the Scade model:
\begin{enumerate} 
\item \label{s0-case}In the specification, \texttt{S0} state has no associated
  requirement other than the \texttt{E015} requirement. Considering
  only the flowchart, \texttt{S0} is an initial state which can be
  left when the conditions explained in \texttt{E015} requirement are
  fulfilled. Usually, states are not conditions.  The Scade model is
  considering \texttt{S0} state as a re-initialization state:
  \texttt{S0} conditions must be fulfilled to allow flowchart
  execution.
%\item The flowchart have end states where no actions are
%  made. Actually, \texttt{A220} and \texttt{A100} (drawn by rounded
%  rectangles) are not states, so they are read one time contrary to a
%  state and then the process (of reading the flowchart) is
%  stopped. This contradicts the Scade model where these elements as
%  seen and modelized as states.
\item \label{a100-case}The flowchart have end states where no actions
  are made. Actually, \texttt{A100} (drawn by rounded rectangles) is
  not a state, so it is read one time contrary to a state and then the
  process (of reading the flowchart) is stopped. This contradicts the
  Scade model where this element is seen and modelized as a state.
\item \label{a220-case}In the case of \texttt{A220}, the flowchart leads to an end
  state after this action. However, the Scade model leads to the
  initial state (a state just before \texttt{E015}).
\item \label{eom-case}The \texttt{A100} action is specified by a requirement which
  says that the process shall go to the ``End of Mission''
  procedure. In the Scade model, we assume this requirement is
  corresponding to the output called
  \texttt{End\_Of\_Mission\_Procedure\_Req}. This output shall be
  \texttt{True} when conditions to reach \texttt{A100} are fulfilled,
  \texttt{False} in all other cases. However this output is equal to
  the input \texttt{On\_Going\_Mission} and, because of the previous
  point, this implies that output
  \texttt{End\_Of\_Mission\_Procedure\_Req} can have a changing value
  when Scade model is in a dead end state where conditions to reach
  \texttt{A100} were fulfilled.
\end{enumerate} 

Here is propositions of solutions to resolve previous problems:
\begin{description}
\item[Proposition for \ref{s0-case}.] The ambiguity of \texttt{S0} definition must be
  clarified: either it is a state with a clear action in its
  definition, or it is a condition which trigger or not the procedure.
\item[Proposition for \ref{a220-case}.] To be equivalent to the
  specification, the Scade model should end in a dead end state
  instead of going back to its initial state.
\item[Proposition for \ref{a100-case}. and \ref{eom-case}.] To be
  equivalent to the specification, the Scade model should not have
  dead end states where actions can be taken (outputs are set to
  values). Actions in these dead end state should be modelized as
  ``emit''.
\end{description}


However, in the case where the Scade model reaches a final state of
the Shunting Initiated By Driver procedure and where it always leads
to a mode change at next cycle, which is not a mode in \texttt{S0}
condition, then the procedure will go back to its initial state after
one cycle in a final state.

So, if this behavior is verified, this will resolve points
\ref{a100-case}. and \ref{eom-case}.

%\paragraph{Constraints used}
%One constraint is used in this model. Without this constraint, a
%counter-example is reachable.

%This constraint specifies that the flowchart shall return to his
%initial state a cycle after the dead-end state \texttt{A100} is
%reached.

%Actually, if this constraint is not verified, the Scade model could
%return \texttt{False} for the request of ``End of Mission'' procedure
%which contradicts the flowchart (\texttt{D040} and \texttt{A100}
%requirements).

%In our implementation of the flowchart, this constraint is modelised
%by verifying that when we are in \texttt{A100} the request of ``End of
%Mission'' procedure correspond to the value of the input
%\texttt{On-going Mission} as it is modelised in the Scade model.

%This modelisation is justified by the fact that, according to
%\texttt{A050}, \texttt{D040} and \texttt{A100}, if the input
%\texttt{On-going Mission} is \texttt{True} then the ``End of Mission''
%request is \texttt{True}, so equal to \texttt{On-going Mission}. Also,
%as transition to SH mode is enable (\texttt{A050}) when ``End of
%Mission'' request is made, the system should go to SH mode (or another
%mode except SB mode). So the next cycle the driver should not execute
%the shunting selection procedure and the next time this pocedure is
%executed it should start at the initial state.


