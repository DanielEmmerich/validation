\section{Methodology for flowchart verification}
\label{methodology}
This section presents the methodology used to verify that the SRS 5 is
correctly implemented into a scade model.

\subsection{Overview of proof methodology}
\label{overview}
The document D1 is a specification written with flowcharts. A scade
model is made according to this specification.

We would like to known that the scade model is a correct
implementation of the specification. To do so, the scade model is
proven equivalent to its flowcharts.

Starting with one flowchart, a first step is to identify to which
scade inputs and outputs this flowchart refers to. Usually, there are
states which send data (scade model outputs) and there are conditions
depending on variable(s) value(s) (scade model inputs).

Then, the flowchart is translated into an equivalent HLL model using
the scade inputs names. Scade outputs names are used to write proof
obligation.

Once these steps done, it is possible to prove (or disprove) that for
all possible inputs, the scade model and the flowchart HLL model
compute the same outputs.


\subsection{Flowchart to HLL model}
\label{flowchart-2-hll}
In this section, a presentation of flowchart (from the SRS
documentation) transformation into an equivalent HLL model is
made. Section~\ref{flowchart-definition} presents definitions of
flowchart shapes, the section~\ref{reading-a-flowchart} explains how a
flowchart is interpreted in the context of the synchronous language
HLL and the section~\ref{transformation-to-hll-program} shows a
generic methodology for the translation of this type of flowchart into
a HLL model.
\subsubsection{Flowcharts in SRS documentation}
\label{flowchart-definition}
Here is a list of flowchart shapes and there meaning:
\begin{description}
\item[Rectangle] The rectangle usually means that an action has to be
  taken. In our case, this shape is assimilated to a state where a
  value can be send through one or more outputs.
\item[Rounded rectangle] This shape also means that an action has to
  be taken, like rectangle shape, but rounded rectangle are read
  differently. We will call them action.
\item[Diamond] This shape usually corresponds to a question and the
  answer indicates the reading direction. Answers are listed on
  different arrows leaving the diamond.
\item[Arrow] This shape indicates the reading direction. An arrow
  should start from a shape and end to another shape. The arrow may be
  labeled. This label is usually a condition which must be fulfilled
  to continue to the next shape.
\item[Circle] This is usually a terminal shape, which indicates an end
  of the flowchart. No more actions will be taken.
\end{description}

Also, a definition of states (which correspond to rectangles and
rounded rectangles) is given in the SRS at section 5.3.2.3.

A definition of transitions (which correspond to diamonds and arrows)
is given in the SRS at section 5.3.2.4.

\subsubsection{Reading a flowchart in a HLL context}
\label{reading-a-flowchart}
The concept of cycle is used in the synchronous language HLL and this
impacts how a flowchart is read when we want to translate it into a
HLL model.

At the first cycle, we start to read the flowchart from an initial
state which is a state where there is only leaving arrows. Then, at
each cycle we go from one state A to a state B, where A could be equal
to B, until we reach a final state which is a state where there is no
leaving arrows.

When, during a cycle, the reading go through an action (rounded
rectangle) then its coresponding actions are taken and the reading
continue. So it is not possible to stop the reading in an action. 

During a cycle execution it is not possible to read more than one
state, so when reaching a state the reading is stopped even if it is
possible to leave the current state following arrows. 

Moreover, initial states can sometimes behave as a global condition
which trigger or stop (and re-initialize) the reading of the
flowchart. In this case, this initial state will not be seen as a
state as mentioned above. Reading corresponding description of the
requirement should help in state interpretation. 


\subsubsection{Transformation to HLL model}
\label{transformation-to-hll-program}
In this section, we will describe how to translate examples of simple
flowcharts into a HLL models. Combining these examples is a way to
translate more complex flowcharts.

In this document, there is only a few examples of flowcharts
translation. An objective can be to automatize this translation.


\subsubsection{Example: Rectangle -Labeled arrow-> Rectangle -Labeled arrow-> Rectangle.}
\begin{figure}[h]
\centering
\begin{tikzpicture}[%
    >=triangle 60,              % Nice arrows; your taste may be different
    start chain=going below,    % General flow is top-to-bottom
    node distance=6mm and 60mm, % Global setup of box spacing
    ]
% ------------------------------------------------- 
% A few box styles 
% <on chain> *and* <on grid> reduce the need for manual relative
% positioning of nodes
\tikzset{
  base/.style={draw, on chain, on grid, align=center, minimum height=4ex},
  state/.style={base, rectangle, text width=8em},
  % -------------------------------------------------
  % Connector line styles for different parts of the diagram
  transition_label/.style ={base, rectangle, draw=none, thick, fill=none}
}

\node [state] (s0) {State 0};
\node [transition_label] (ta) {Condition A};
\node [state] (s1) {State 1};
\node [transition_label] (tb) {Condition B};
\node [state] (s2) {State 2};


\draw [-] (s0.south) -- (ta);
\draw [->] (ta.south) -- (s1);
\draw [-] (s1.south) -- (tb);
\draw [->] (tb.south) -- (s2);

\end{tikzpicture}
\caption{Three states and two transitions.}
\label{example-1}
\end{figure}


The following HLL model is translation of the flowchart
figure~\ref{example-1}: 
{\footnotesize
\begin{verbatim}
Types:
enum{'state_0','state_1','state_2'} states;

Declarations:
states last_states; // State reached at previous cycle
states current_states; // Current state

// Transitions:
states transition_A;
states transition_B;

/* Declarations of actions took (= output(s) are set to a value) for
 each state: */
bool s_0_outputs;
bool s_1_outputs;
bool s_2_outputs;

/* Main proof obligation */
bool po;

Definitions:
s_0_outputs := State 0;
s_1_outputs := State 1;
s_2_outputs := State 2;

// Definition of transitions between states:
last_state := pre(current_state,'state_0'); // 'state_0' is an initial state
current_state := (last_state
                 | 'state_0' => transition_A
                 | 'state_1' => transition_B
                 | 'state_2' => 'state_2' // 'state_2' is a final state
                 );

transition_A := if (Condition A) then
                'state_1'
                else
                'state_0';

transition_B := if (Condition B) then
                'state_2'
                else
                'state_1';


/* Main proof obligation */
po := if (current_state = 'state_0') then s_0_outputs
      elif (current_state = 'state_1') then s_1_outputs
      elif (current_state = 'state_2') then s_2_outputs
      else False;

Proof Obligations:
po;
\end{verbatim}
}


\subsubsection{Example: Rectangle -Labeled arrow-> Rounded rectangle --> Diamond ->> Rectangle x 2.}
\begin{figure}[h]
\centering
\begin{tikzpicture}[%
    >=triangle 60,              % Nice arrows; your taste may be different
    start chain=going below,    % General flow is top-to-bottom
    node distance=6mm and 60mm, % Global setup of box spacing
    ]
% ------------------------------------------------- 
% A few box styles 
% <on chain> *and* <on grid> reduce the need for manual relative
% positioning of nodes
\tikzset{
  base/.style={draw, on chain, on grid, align=center, minimum height=4ex},
  state/.style={base, rectangle, text width=8em},
  test/.style={base, diamond, aspect=2, text width=5em},
  tmp_state/.style={state, rounded corners},
  % -------------------------------------------------
  % Connector line styles for different parts of the diagram
  transition_label/.style ={base, rectangle, draw=none, thick, fill=none}
}

\node [state] (s0) {State 0};
\node [transition_label] (ta) {Condition A};
\node [tmp_state] (is1) {Action 1};
\node [test] (d) {Question ?};
\node [transition_label] (d1) {Answer 1};
\node [state] (s2) {State 2};
\node [transition_label] (d2) [right=of d] {Answer 2};
\node [state] (s3) {State 3};

\draw [-] (s0.south) -- (ta);
\draw [->] (ta.south) -- (is1);
\draw [->] (is1.south) -- (d);
\draw [-] (d.south) -- (d1);
\draw [->] (d1.south) -- (s2);
\draw [-] (d.east) -- (d2);
\draw [->] (d2.south) -- (s3);

\end{tikzpicture}
\caption{Three states, one rounded rectangle state, one diamond and one transition.}
\label{example-2}
\end{figure}

The following HLL model is translation of the flowchart
figure~\ref{example-2}: 
{\footnotesize
\begin{verbatim}
Types:
enum{'state_0','state_2','state_3'} states;

Declarations:
states last_states; // State reached at previous cycle
states current_states; // Current state

// Transitions:
states transition_A;
states transition_question;

/* Declarations of actions took (= output(s) are set to a value) for
 each state: */
bool state_0_outputs;
bool state_2_outputs;
bool state_3_outputs;

// Activation state of rounded rectangle states
bool action_1_activation;


/* Main proof obligation */
bool po;

Definitions:
state_0_outputs := State 0;
state_0_action_1_outputs := State 0 & Action 1;
state_2_outputs := State 2;
state_2_action_1_outputs := State 2 & Action 1;
state_3_outputs := State 3;
state_3_action_1_outputs := State 3 & Action 1;

// Definition of transitions between states:
last_state := pre(current_state,'state_0'); // 'state_0' is an initial state
current_state := (last_state
                 | 'state_0' => transition_A
                 | 'state_2' => 'state_2' // 'state_2' is a final state
                 | 'state_3' => 'state_3' // 'state_3' is a final state
                 );

transition_A := if (Condition A) then
                transition_question
                else
                'state_0';

transition_question := if (Answer 1) then
                       'state_2'
                       elif (Answer 2) then
                       'state_3'
                       else
                       'state_0';

action_1_activation := (last_state = 'state_0') & Condition A;

/* Main proof obligation */
po := if (current_state = 'state_0' & ~action_1_activation) then state_0_outputs
      elif (current_state = 'state_0' & action_1_activation) then state_0_action_1_outputs
      elif (current_state = 'state_2' & ~action_1_activation) then state_2_outputs
      elif (current_state = 'state_2' & action_1_activation) then state_2_action_1_outputs
      elif (current_state = 'state_3' & ~action_1_activation) then state_3_outputs
      elif (current_state = 'state_3' & action_1_activation) then state_3_action_1_outputs
      else False;

Proof Obligations:
po;
\end{verbatim}
}


\subsection{Equivalence proof between flowchart and Scade model}
We want to prove that our flowchart take the same actions than the
Scade model for each flowchart state.

To do this, three elements are required to construct the proof
obligation:
\begin{enumerate}
\item State names, usualy they are listed in an enumerated type.
\item Action (corresponding to rounded rectangle) activation
  conditions. They are implemented by boolean variables which are
  \texttt{True} when the action is read during a cycle execution,
  \texttt{False} otherwise.
\item Boolean variables representing outputs values for each
  states. Each Scade model outputs must be present in each variable.
\end{enumerate}

Once these three elements are implemented, the proof obligation can be
constructed using an \texttt{if ... then ... elif ... else ...;}
structure as we can see in previous examples.

This structure will represent a following statement:
\begin{verbatim}
if "flowchart is in a state (and if action(s) is taken (or not))"
then "Scade outputs are set to some values" 
elif ...
\end{verbatim}

The \texttt{... else False;} part is a security: if we forgot a case the
proof obligation will be \texttt{False} and our attention will be
caught on this problem.

\paragraph{}
With this proof obligation, S3 tool will be able to prove or disprove
(or fail to prove) that, at each cycle, actions taken by a flowchart
HLL model are the same as Scade model.
