\section{Methodology for flowchart verification}
\label{methodology}
This section presents the methodology used to verify that the SRS 5 is
correctly implemented into a scade model.

\subsection{Overview of proof methodology}
\label{overview}
The document D1 is a specification written with flowcharts. A scade
model is made according to this specification.

We want to known that the scade model is a correct implementation of
the specification. To do so, we will prove that the scade model
is equivalent to flowcharts.


Starting with one flowchart, a first step is to identify to which
scade inputs and outputs this flowchart refers to. Usually, there are
states which send data (scade model outputs) and there are conditions
depending on variable(s) value(s) (scade model inputs).

Then, the flowchart is transalted into an equivalent HLL program using
the scade inputs names. Scade outputs names are used to write proof
obligation.

Once those steps done, it is possible to prove (or disprove) that for
all possible inputs, the scade model and the flowcart HLL program
compute the same outputs.


\subsection{Flowchart to HLL program}
\label{flowchart-2-hll}
In this section we will see how to transform a flowchart from the SRS
documentation into an equivalent HLL
program. Section~\ref{flowchart-definition} presents definitions of
flowchart shapes, the section~\ref{reading-a-flowchart} explains how a
flowchart is interpreted in the context of the synchronous language
HLL and the section~\ref{transformation-to-hll-program} shows a
generic methodology for the translation of this type of flowchart into
a HLL program.
\subsubsection{Flowcharts in SRS documentation}
\label{flowchart-definition}
Here is a list of flowchart shapes and there meaning :
\begin{description}
\item[Rectangle] The rectangle usually means that an action has to be
  taken. In our case, this shape is assimilated to a state where a
  value can be send through one or more outputs.
\item[Rounded rectangle] This shape is equal to the rectangle shape
  except when you read it (see section~\ref{reading-a-flowchart}). If
  no arrow is leaving from such shape, it is considered as a
  rectangle.
\item[Diamond] This shape usually corresponds to a question and the
  answer indicates the reading direction. Answers are listed on
  different arrows leaving the diamond.
\item[Arrow] This shape indicates the reading direction. An arrow
  should start from a shape and end to another shape. The arrow may be
  labeled. This label is usually a condition which must be fulfilled
  to continue to the next shape.
\item[Circle] This is usually a terminal shape, which indicates an end
  of the flowchart. No more actions will be taken.
\end{description}

Also, a definition of states (which correspond to rectangles and
rounded rectangles) is given in the SRS at section 5.3.2.3.

A definition of transitions (which correspond to diamonds and arrows)
is given in the SRS at section 5.3.2.4.

\subsubsection{Reading a flowchart in a HLL context}
\label{reading-a-flowchart}
The concept of cycle is used in the synchronous language HLL and this
impacts how a flowchart is read when we want to translate it into a
HLL program.

At the first cycle, we start to read the flowchart from an initial
state which is a state where there is only leaving arrows. Then, at
each cycle we go from one state A to a state B, where A could be equal
to B, until we reach a final state which is a state where there is no
leaving arrows.

During a cycle execution it is not possible to read more than one
state, so when reaching a state the reading is stopped even if it is
possible to leave the current state following arrows. 

Moreover, intial states can sometimes behave as a global condition
which trigger or stop (and re-initialize) the reading of the
flowchart. In this case, this initial state will not be seen as a
state as mentionned above. Reading corresponding description of the
requirement should help in state interpretation.


\subsubsection{Transformation to HLL program}
\label{transformation-to-hll-program}
In this section, we will describe how to translate examples of simples
flowcharts into a HLL programs. Combining those examples is a way to
translate more complex flowcharts. 

\begin{figure}[h]
\centering
\begin{tikzpicture}[%
    >=triangle 60,              % Nice arrows; your taste may be different
    start chain=going below,    % General flow is top-to-bottom
    node distance=6mm and 60mm, % Global setup of box spacing
    ]
% ------------------------------------------------- 
% A few box styles 
% <on chain> *and* <on grid> reduce the need for manual relative
% positioning of nodes
\tikzset{
  base/.style={draw, on chain, on grid, align=center, minimum height=4ex},
  state/.style={base, rectangle, text width=8em},
  % -------------------------------------------------
  % Connector line styles for different parts of the diagram
  transition_label/.style ={base, rectangle, draw=none, thick, fill=none}
}

\node [state] (s0) {State 0};
\node [transition_label] (ta) {Condition A};
\node [state] (s1) {State 1};
\node [transition_label] (tb) {Condition B};
\node [state] (s2) {State 2};


\draw [-] (s0.south) -- (ta);
\draw [->] (ta.south) -- (s1);
\draw [-] (s1.south) -- (tb);
\draw [->] (tb.south) -- (s2);

\end{tikzpicture}
\caption{Three states and two transitions.}
\label{example-1}
\end{figure}


\paragraph{Rectangle -Labeled arrow-> Rectangle -Labeled arrow-> Rectangle.}
The following HLL program is translation of the flowchart
figure~\ref{example-1} : 
{\footnotesize
\begin{verbatim}
Types:
enum{'state_0','state_1','state_2'} states;

Declarations:
states last_states; // State reached at previous cycle
states current_states; // Current state

states s_0;
states s_1;

bool s_0_var_state;
bool s_1_var_state;
bool s_2_var_state;

Definitions:
s_0_var_state := State 0;
s_1_var_state := State 1;
s_2_var_state := State 2;

last_state := pre(current_state,'state_0'); // 'state_0' is an initial state
current_state := (last_state
                 | 'state_0' => s_0
                 | 'state_1' => s_1
                 | 'state_2' => 'state_2' // 'state_2' is a final state
                 );

s_0 := if (Condition A) then
       'state_1'
       else
       'state_0';

s_1 := if (Condition B) then
       'state_2'
       else
       'state_1';

Proof Obligations:
(current_state = 'state_0') <-> s_0_var_state;
(current_state = 'state_1') <-> s_1_var_state;
(current_state = 'state_2') <-> s_2_var_state;
\end{verbatim}
}


\begin{figure}[h]
\centering
\begin{tikzpicture}[%
    >=triangle 60,              % Nice arrows; your taste may be different
    start chain=going below,    % General flow is top-to-bottom
    node distance=6mm and 60mm, % Global setup of box spacing
    ]
% ------------------------------------------------- 
% A few box styles 
% <on chain> *and* <on grid> reduce the need for manual relative
% positioning of nodes
\tikzset{
  base/.style={draw, on chain, on grid, align=center, minimum height=4ex},
  state/.style={base, rectangle, text width=8em},
  test/.style={base, diamond, aspect=2, text width=5em},
  tmp_state/.style={state, rounded corners},
  % -------------------------------------------------
  % Connector line styles for different parts of the diagram
  transition_label/.style ={base, rectangle, draw=none, thick, fill=none}
}

\node [state] (s0) {State 0};
\node [transition_label] (ta) {Condition A};
\node [tmp_state] (is1) {Intermediate State 1};
\node [test] (d) {Question ?};
\node [transition_label] (d1) {Answer 1};
\node [state] (s2) {State 2};
\node [transition_label] (d2) [right=of d] {Answer 2};
\node [state] (s3) {State 3};

\draw [-] (s0.south) -- (ta);
\draw [->] (ta.south) -- (is1);
\draw [->] (is1.south) -- (d);
\draw [-] (d.south) -- (d1);
\draw [->] (d1.south) -- (s2);
\draw [-] (d.east) -- (d2);
\draw [->] (d2.south) -- (s3);

\end{tikzpicture}
\caption{Three states and two transitions.}
\label{example-2}
\end{figure}
\paragraph{Rectangle -Labeled arrow-> Rounded rectangle --> Diamond ->> Rectangle x 2.}
The following HLL program is translation of the flowchart
figure~\ref{example-2} : 
{\footnotesize
\begin{verbatim}
Types:
enum{'state_0','inter_state_1','state_2','state_3',
     'state_0_from_diamond','state_2_from_state_1','state_3_from_state_1'} states;

Declarations:
states last_states; // State reached at previous cycle
states current_states; // Current state

states s_0;
states inter_s_1;

bool s_0_var_state;
bool s_2_var_state;
bool s_3_var_state;
bool inter_s_1_state;

Definitions:
s_0_var_state := State 0;
s_2_var_state := State 2;
s_3_var_state := State 3;
inter_s_1_state := Inter_State 1;

last_state := pre(current_state,'state_0'); // 'state_0' is an initial state
current_state := (last_state
                 | 'state_0' => s_0
                 | 'state_0_from_diamond' => s_0
                 | 'state_2_from_state_1' => 'state_2'
                 | 'state_3_from_state_1' => 'state_3'
                 | 'state_2' => 'state_2' // 'state_2' is a final state
                 | 'state_3' => 'state_3' // 'state_3' is a final state
                 );

s_0 := if (Condition A) then
       inter_s_1
       else
       'state_0';

inter_s_1 := if (Diamond_cond_1) then
             'state_2_from_state_1'
             elif (Diamond_cond_2) then
             'state_3_from_state_1'
             else
             'state_0_from_diamond';


Proof Obligations:
(current_state = 'state_0') <-> s_0_var_state;
(current_state = 'state_0_from_diamond') <-> s_0_var_state & inter_s_1_state;
(current_state = 'state_2_from_state_1') <-> s_2_var_state & inter_s_1_state;
(current_state = 'state_3_from_state_1') <-> s_3_var_state & inter_s_1_state;
(current_state = 'state_2') <-> s_2_var_state;
(current_state = 'state_3') <-> s_3_var_state;
\end{verbatim}
}

Here is only a few examples of flowcharts translation. An objective
can be to automatize this translation.




\subsection{Proofs that a flowchart corresponds to its model}

In HLL programs translated from~\ref{example-1} and ~\ref{example-2},
we can see that proof obligations are equivalences between
states. Usually, right hand side of the equivalence are streams
composed with Scade model outputs. 
