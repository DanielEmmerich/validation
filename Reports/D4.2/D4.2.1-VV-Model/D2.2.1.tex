\documentclass{template/openetcs_article}
% Use the option "nocc" if the document is not licensed under Creative Commons
%\documentclass[nocc]{template/openetcs_article} 
\usepackage{lipsum,url}
\usepackage{xspace}
\usepackage{graphicx}
\usepackage{fixme}
\usepackage{lscape} 
\usepackage{pgfgantt}
\usepackage{adjustbox}
\usepackage{datetime}



%user specified macros


\newcommand{\VV}{Verification \& Validation\xspace}
\newcommand{\vv}{verification \& validation\xspace}

\def\CC{{C\nolinebreak[4]\hspace{-.05em}\raisebox{.4ex}{\tiny\bf ++}}}

\newcommand{\bitwalker}{\mbox{\texttt{Bitwalker}}\xspace}

\newcommand{\poke}{\mbox{\texttt{Bitwalker\_Poke}}\xspace}
\newcommand{\peek}{\mbox{\texttt{Bitwalker\_Peek}}\xspace}
\newcommand{\acsl}{\mbox{\textsf{ACSL}}\xspace}
\newcommand{\isoc}{\mbox{\textsf{C}}\xspace}
\newcommand{\framac}{\mbox{\textsf{Frama-C}}\xspace}
\newcommand{\framacwp}{\mbox{\textsf{Frama-C\slash WP}}\xspace}
\newcommand{\why}{\mbox{\textsf{Why}}\xspace}
\newcommand{\wpframac}{\mbox{\textsf{WP}}\xspace}
\newcommand{\altergo}{\mbox{\textsf{Alt-Ergo}}\xspace}
\newcommand{\qed}{\mbox{\textsf{Qed}}\xspace}
\newcommand{\cvc}{\mbox{\textsf{CVC4}}\xspace}
\newcommand{\z}{\mbox{\textsf{Z3}}\xspace}
\newcommand{\coq}{\mbox{\textsf{Coq}}\xspace}
\newcommand{\cealist}{\mbox{\textsf{CEA LIST}}\xspace}

\newcommand{\inl}[1]{\lstinline[style=inline]{#1}}




\graphicspath{{./template/}{.}{./images/}}
\begin{document}
\frontmatter
\project{openETCS}

%Please do not change anything above this line
%============================
% The document metadata is defined below

\reportnum{OETCS/WP4/D4.2.2}

%define your workpackage or task here
\wp{openETCS@ITEA Work Package 4.2: ``Verification \& Validation of the Formal Model''}

%set a title here
\title{D.4.2.2 1st interim V\&V report on the applicability of the V\&V approach to the formal abstract model}

%set a subtitle here
\subtitle{}

%set the date of the report here
\date{October 2013}

%define a list of authors and their affiliation here
\author{Ana Cavalli \and João Santos}

\affiliation{Télécom SudParis\\
  9 rue Charles Fourier\\
  91011 Evry Cedex, France}
  
% define the coverart
\coverart[width=350pt]{openETCS_EUPL}

%define the type of report
\reporttype{}



\begin{abstract}
%define an abstract here

\lipsum[12-13]

\end{abstract}

%=============================
%Do not change the next three lines
\maketitle
\tableofcontents
\listoffiguresandtables
\newpage
%=============================

% The actual document starts below this line
%=============================


%Start here

\section*{Introduction}

To ensure the correctness and consistency of the model and its implementation, the validation
and verification has to be performed alongside with the modelling process. Thus these tasks will
be performed repeatedly during WP3 and will provide feedback to it.

This document presents the results of the first iteration of verification and validation of the formal model. This was be accomplished
by applying the methods chosen in WP4 Task 1 onto the formal model using the tool
chain developed in WP7. 

\section{Model Verification Techniques}

\subsection{Proof Technique}

A proof is a demonstration that if some fundamental statements (axioms) are assumed to be
true, then some mathematical statement is necessarily true. As mentioned in the requirements
document produced by WP2, as much as possible, formal proof would then be used to prove
that the OpenETCS model never enter a Feared State, as long as the other subsystem (RBC,
communication layer. . . ) fulfill their own safety properties (axiom describing the environment).
Such theorem proving helps to increase our confidence on the specified model. The proof
techniques should be integrated in the selected tool chain.
In order to use formal proof to verify if the SFM (Semi-formal model) and FFM (fully formal
model) comply with the safety and function requirements (cf. R-WP2/D2.6-02-058), the properties
to be proven have to be identified and described. There will be a set of axioms that will
describe both functional and/or safety properties of the system. The choice of axioms describing
functional and/or safety properties will be provided by safety analysis in an independent way
from approaches used to specify, design, validate or verify. It must be noted that the model
obtained from the Subsystem Requirements Specification should be verified in this manner at a
first stage.

\subsection{Model Checking}

Model checking is an automatic technique for verifying finite-state reactive systems. As such,
one could automatically check if the model specifies most of the requirements of the system,
such as the important safety properties described in Task 4.4.
Similar to proof techniques, in order to use model checking to verify if the SFM (Semi-formal
model) and FFM (fully formal model) comply with the safety and function requirements (cf. RWP2
/D2.6-02-058), the properties to be proven have to be identified and described. To implement the use model checking, it is mandatory to specify the model using finite-state reactive systems,
and they should also provide an intuitive way to express the properties to be model checked. The
set of critical requirements to be verified need to be clearly identified. The criteria for the model
to be considered a representation of the standard is that all properties are checked. The proposed
model checking techniques should be supported in the selected tool chain.

**Should include the list of properties to be evaluated by model checking

\subsection{Simulation}

As for simulation, the artifacts should provide means to execute the model. The simulator must be
automatically generated, so that, when run against test scenarios (inputs/outputs for the model),
we may conclude whether the model follows the specification or not. In particular, it is important
to define test scenarios for the safety critical properties. Since, the development within openETCS
has to the goal to reach the CENELEC EN 50128 SIL 4 standard, it is highly recommended (cf.
SIL 4) that the simulation needs to cover all states, transitions, data-flow, and paths in the model.
It would also be desirable to include graphical representation of the simulation/model and also
provide a report of the visited components as specified by CENELEC EN 50128 SIL 4.
CENELEC EN 50128 SIL4 also advocates to perform tracing. Being able to trace the requirements
that are met during a simulation is also advisable to allow simple requirement coverage.

**Should include the list of properties to be evaluated by simulation

\subsection{Other Techniques}

Reviews, Inspections, static analysis and walkthroughs, mostly manual techniques, are also to be
considered for the verification of models.

**To be completed with the techniques proposed by the partners. 

\section{Tools used for the Verification of the model}

The tools used with these techniques are described in WP7.

\section{Applied Techniques}

This section will describe results from V\&V on each of the categories. Categories for other techniques proposed by partners should be added.

Each used technique should presented as follows

\begin{itemize}
\item Description of the applied technique
\item Description of the model
\item Description of the results
\end{itemize}

\subsection{Proof Technique}

Results obtained by applying proof techniques

\subsection{Model Checking}

Results obtained by applying Model Checking

\subsection{Simulation}

Results obtained by applying simulation

\subsection{Other Techniques}

\section*{Conclusions}

Final deliberation on the results presented on the previous chapters
\nocite{*}
%===================================================
%Do NOT change anything below this line

\end{document}
