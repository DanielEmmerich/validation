
\section{Introduction}

While major parts of the functionality of {Subset 026} are developed in 
higher-level languages, there is also a substantial part of \emph{supporting} software
that is developed in the programming language~C.

In this document we report about \emph{preliminary} results on the verification
of C-code developed in the OpenETCS project.
In particular, we report on the use of static analysis methods (including formal methods)
on C code that has been developed by the project partner Siemens (Germany).

This software has been analyzed by the OpenETCS project partners SQS (Spain)
and Fraunhofer FOKUS (Germany).
The Frama-C tool, which is developed by the French project partner {CEA LIST},
has been used for some of the analyses.

In Section~\ref{sec:sqs} we report about the results of a broad range
of static analyses. 
These methods are aimed at finding well-known deficiencies that might occur in C or \CC\ software.

In Section~\ref{sec:fokus} we take a more formal approach by 
\begin{enumerate}
\item formally specifying the expected functional behavior in the ACSL specification language of {Frama-C}
      and
\item using the {Frama-C} verification platform to establish a formal proof that the C code
      satisfies the formal specification.
\end{enumerate}

Regarding the more formal approach it must be pointed out that so far only a 
part of Siemens' \emph{BitWalker} has been formalized and verified.
In the process of this work several enhancements for the Frama-C verification platform have 
been identified and reported to the developers at {CEA LIST}.




