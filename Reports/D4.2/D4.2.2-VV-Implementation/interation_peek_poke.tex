
\section{Interaction of \peek and \poke}
\label{interaction}


In this section we examine the interaction of 
\peek and \poke. The functions shall be inverse to each other
with respect to the normal cases of both functions.
In the following we provide a formal specification and present
our verification results.


\section{Formal Specification with ACSL and C}

For the specification of the interaction we have two auxiliary \isoc functions:
The first one to call first \poke on a bit stream and then \peek
and the second one to do it the other way around.

Figure~\ref{fig:poke-peek} shows the straightforward implementation 
of the first helper function along with its \acsl contract.
The contract contains a lot of preconditions because we
only specify an interaction for the case where both 
functions are called in their normal
cases. The reader can compare the preconditions with the ones from
the contracts of \peek and \poke with respect to the \inl{normal} behaviors.
As a postcondition we formulate that \peek reads exactly the value
written by \poke.

\begin{listing}[hbt]
\begin{minipage}{\textwidth}
\lstinputlisting[style=acsl-block, frame=single, linerange=3-20]{./figures/Peek_Poke_inverse.c}
\end{minipage}
\caption{\label{fig:poke-peek} Specification of interaction when first calling \poke.}
\end{listing}

Figure~\ref{fig:peek-poke} shows the straightforward implementation 
of the second auxiliary function along with its \acsl contract.
In contrast to the first function, we work with two bit streams.
With \peek a certain bit sequence is read out from one bit stream
and written via \poke into the other one.
Therefore, we formulate as a postconditions that
the two bit streams are equal in these certain ranges.

\begin{listing}[hbt]
\begin{minipage}{\textwidth}
\lstinputlisting[style=acsl-block, frame=single, linerange=22-43]{./figures/Peek_Poke_inverse.c}
\end{minipage}
\caption{\label{fig:peek-poke} Specification of interaction when first calling \peek.}
\end{listing}

 \FloatBarrier

\section{Formal Verification with Frama-C/WP}

All postconditions (and the assertion in figure~\ref{fig:peek-poke})
are verified. Therefore, we succeeded to prove that the 
functions \peek and \poke interact correctly
with respect to their specifications.

Moreover, this verification result serves as a validation for the contracts
of \peek and \poke because verification of the interaction 
depends on these contracts. Since both functions are called,
the caller has to ensure that all preconditions hold
and can then rely on the guarantees given by the postconditions.

As a remark, we extended the contract of \peek by one postcondition 
to ease the verification of the interaction.
The postcondition simply states that the read value is always representable 
by \inl{Length} bits which is obviously always the case,
since the value is the sum of \inl{Length} bits.
The postcondition is needed to ensure that the preconditions for the normal case
of \poke are fulfilled when first calling \peek and then \poke.

\clearpage

