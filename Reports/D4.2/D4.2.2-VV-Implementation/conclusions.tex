
\chapter{Conclusions}
\label{sec:conclusions}

This report presents experiments with various static analysis techniques:

\begin{enumerate}
\item formal verification with \framacwp (see Chapters~\ref{sec:frama-c}
      and~\ref{sec:formal-verification})
\item static analysis methods (see Chapter~\ref{sec:static-analysis})
\end{enumerate}


Deductive verification with \framacwp allows to prove with mathematical strength
that software satisfies it functional specification.
This approach has been applied to railway software \cite{INDIN2103-FOKUS}
and other software components\cite{thesisKimVoellinger} before.
The technical challenge for the software analyzed in the report at hand is
the extensive use of low-level bit operations.

Verification with \framacwp is intended mostly for the level of software components.
Thus, one can imagine that this approach could replace tests on the 
level of software components but surely not for software integration
and software\slash hardware integration.

Moreover, deductive verification with \framacwp works most efficiently
if there is already a sufficiently precise \emph{informal} specification.
This was not the case for the \bitwalker which only was accompanied with
a high-level description of its intended functionality.
However, neither the admissible inputs nor the expected results (including error conditions)
were properly described.
Instead, this information had to be discovered from the implementation and cross checked
with the software designer.

Since formal verification is a very precise but also relatively expensive technique,
it does not make sense to use it without using cheaper verification techniques first.
Specifically, we highly recommend  simpler static analyses from 
Chapter~\ref{sec:static-analysis} to quickly identify code deficiencies
and to fix them \emph{before} a tool like \framacwp is applied.

