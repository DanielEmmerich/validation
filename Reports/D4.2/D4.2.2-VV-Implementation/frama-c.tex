
\section{An Introduction to Formal Verification with Frama-C\slash\ WP}
\label{sec:frama-c}

Frama-C is platform dedicated to source-code analysis of C software.
It has a plug-in architecture and can thus be easily extended to 
different kinds of analyses.
The WP plugin of Frama-C allows to formally verify that a a piece of
C code satisfies its specification.
This implies, of course, that the user provides a \emph{formal specification}
of what the implementation is supposed to do.
Frama-C comes with its own specification language ACSL which stands for
\emph{ANSI\slash\ ISO Specification Language}.
In order to help potential users to master ACSL we discuss in this section 
a very simple C function and explain various aspects of ACSL.

\subsection{First steps}

\begin{listing}[hbt]
\begin{minipage}{\textwidth}
\lstinputlisting[style=acsl-block ]{./Abs/abs.c}
\end{minipage}
\caption{\label{fig:abs} An implementation of the absolute value function}
\end{listing}

\begin{listing}[hbt]
\begin{minipage}{\textwidth}
\lstinputlisting[style=acsl-block ]{./Abs/abs1.c}
\end{minipage}
\caption{\label{fig:abs1} A first attempt to formally specify \inl{abs_int}}
\end{listing}

\begin{listing}[hbt]
\begin{minipage}{\textwidth}
\lstinputlisting[style=acsl-block ]{./Abs/test_abs.c}
\end{minipage}
\caption{\label{fig:test_abs} Some simple test cases}
\end{listing}

\begin{listing}[hbt]
\begin{minipage}{\textwidth}
\lstinputlisting[style=acsl-block ]{./Abs/abs1a.c}
\end{minipage}
\caption{\label{fig:abs1a} Taking integer overflows into account}
\end{listing}

\FloatBarrier

\subsection{Separating specification and implementation}

\begin{listing}[hbt]
\begin{minipage}{\textwidth}
\lstinputlisting[style=acsl-block ]{./Abs/abs2.h}
\end{minipage}
\caption{\label{fig:abs2-h} Specifying a function prototype}
\end{listing}


\begin{listing}[hbt]
\begin{minipage}{\textwidth}
\lstinputlisting[style=acsl-block ]{./Abs/abs2.c}
\end{minipage}
\caption{\label{fig:abs2-c} Implementation at a different location than the specification}
\end{listing}

\FloatBarrier

\subsection{Modular verification}

\begin{listing}[hbt]
\begin{minipage}{\textwidth}
\lstinputlisting[style=acsl-block ]{./Abs/use_abs2_1.c}
\end{minipage}
\caption{\label{fig:use_abs2-1} A simple example of modular verification}
\end{listing}

\begin{listing}[hbt]
\begin{minipage}{\textwidth}
\lstinputlisting[style=acsl-block ]{./Abs/use_abs2_2.c}
\end{minipage}
\caption{\label{fig:use_abs2-2} Another example of modular verification}
\end{listing}

\begin{listing}[hbt]
\begin{minipage}{\textwidth}
\lstinputlisting[style=acsl-block ]{./Abs/use_abs2_3.c}
\end{minipage}
\caption{\label{fig:use_abs2-3} A more complex example of modular verification}
\end{listing}

\FloatBarrier

\subsection{Dealing with side effects}

\begin{listing}[hbt]
\begin{minipage}{\textwidth}
\lstinputlisting[style=acsl-block ]{./Abs/abs3a.c}
\end{minipage}
\caption{\label{fig:abs3a} An implementation with side effects}
\end{listing}

\begin{listing}[hbt]
\begin{minipage}{\textwidth}
\lstinputlisting[style=acsl-block ]{./Abs/abs3b.c}
\end{minipage}
\caption{\label{fig:abs3b} Specifying the absence of side effects}
\end{listing}

\begin{listing}[hbt]
\begin{minipage}{\textwidth}
\lstinputlisting[style=acsl-block ]{./Abs/abs3c.c}
\end{minipage}
\caption{\label{fig:abs3c} Finer control of side effects}
\end{listing}


\FloatBarrier
