
\subsection{Informal specification of \peek}
\label{sec:informal-peek}

Now we specify \peek based on the introduced auxiliary concepts.
The function \peek reads a bit sequence from a bit stream
and converts it to a 64-bit long integer.

Its function signature reads as follows:

\begin{lstlisting}[style=acsl-block]
uint64_t  Bitwalker_Peek(unsigned int Startposition, 
                         unsigned int Length,
                         uint8_t Bitstream[],
                         unsigned int BitstreamSizeInBytes);
\end{lstlisting}

\paragraph{Arguments}
The arguments of \peek have the following purpose:
\begin{itemize}
    \item \inl{Startposition} is the bit index in the bit stream 
    where the bit sequence starts.
    \item \inl{Length} is the length of the bit sequence.
    \item \inl{Bitstream} is the array which provides the bit stream.
    \item \inl{BitstreamSizeInBytes} is the length of the array 
    containing the bit stream. 
\end{itemize}

\paragraph{Preconditions}
The following preconditions shall hold for the function arguments.
Note that additional constraints are implicitly expressed by the use
of \emph{unsigned} integer types.

\begin{itemize}
\item \inl{Bitstream} is a valid array of length \inl{BitstreamSizeInBytes}

\item \inl{Length} $\leq$ \inl{64} and

\item \inl{Startposition} $\leq$ \inl{UINT_MAX} - \inl{Length}.
      This condition expresses that no arithmetic overflows shall occur
      when evaluating \inl{Startposition + Length}.
\end{itemize}

\paragraph{Description}
As mentioned, the function \peek reads a bit sequence from a bit stream
and converts it to a 64-bit unsigned integer.

For a bit sequence $(b_0, b_1,\ldots,b_{n - 1})$ the function \peek returns the sum
\begin{align}
\label{eq:peek-result}
    \sum_{i=0}^{n-1} b_i \cdot 2^{(n - 1) - i} 
\end{align}

Note that is a higher-level description than what is done in the source code.
There is, in our opinion, not much point to reflect all of the low-level bit operations
into the specification if a clearer description is at hand.

If the bit sequence is not valid, then \peek shall return~0 according
to the Siemens high-level description.
We were wondering why the implementation maps both an illegal input and
a legal one to the same output.
The code providers argued along the lines that this error condition was not
considered important enough to be properly reported.
One can interpret this design decision as an attempt to increase the
robustness of the function against illegal values.
In general, we recommend to explicitly describe all error conditions
and to devise a consistent error detection and error recovery strategy.


\subsection{Informal specification of \poke}
\label{sec:informal-poke}

In this section, we examine the function \poke
in the same manner as we did it for \peek.

The function \poke converts an integer to a bit sequence and writes it
into a bit stream.
Its function signature reads as follows:
\begin{lstlisting}[style = acsl-block]
int      Bitwalker_Poke(unsigned int Startposition,
                        unsigned int Length,
                        uint8_t Bitstream[],
                        unsigned int BitstreamSizeInBytes,
                        uint64_t Value);
\end{lstlisting}


\paragraph{Arguments}
The arguments have the following purpose:

\begin{itemize}
    \item \inl{Startposition} is the bit index in the bit stream 
    where the bit sequence shall start.
    \item \inl{Length} is the length of the bit sequence.
    \item \inl{Bitstream} is the array which provides the bit stream.
    \item \inl{BitstreamSizeInBytes} is the length of the array 
    containing the bit stream. 
    \item \inl{Value} is the integer to be converted into the bit sequence.
\end{itemize}


\paragraph{Preconditions}
The following conditions shall hold for the function arguments:

\begin{itemize}
\item \inl{Bitstream} is a valid array of length \inl{BitstreamSizeInBytes}

\item \inl{Startposition + Length} is less than or equal to \inl{UINT_MAX}.
\end{itemize}

Note that additional constraints are implicitly expressed by the use
of \emph{unsigned} integer types.


\paragraph{Description}
Now we can specify \poke as follows:
The function \poke converts a 64-bit unsigned integer $x$ to a bit sequence and 
writes it into a bit stream.

For $0 \leq x$, there exists a shortest sequence $(b_0, b_1,\ldots,b_{n - 1})$ of~0es and~1s
such that
\begin{align}
    \sum_{i=0}^{n-1} b_i \cdot 2^{(n - 1) - i} = x.
\end{align}

The function \poke tries to store the sequence $(b_0, b_1,\ldots,b_{n - 1})$
in the bit sequence of \inl{Length} bits that starts
at bit index \inl{Startposition}.

The return value of \poke depends on the following three cases:

\begin{itemize}
\item 
If the bit sequence is not valid, then \poke returns~$-1$.

\item 
If the bit sequence is valid, then there are two cases:

\begin{itemize}
\item
If $x$ is greater or equal than $2^\mathtt{Length}$, then~$x$
cannot be represented as bit sequence $(b_0, b_1,\ldots,b_{\mathtt{Length} - 1})$.
\poke returns then~$-2$.

\item
If $x$ is less the $2^{\mathtt{Length}}$, then  the sequence
$(\overbrace{0,\ldots,0}^{\mathtt{Length}-n},b_0, b_1,\ldots,b_{n - 1})$
is stored in the bit stream starting at \inl{Startposition}.
The return value of \poke is 0.

\end{itemize}
\end{itemize}

