\documentclass{template/openetcs_article}
% Use the option "nocc" if the document is not licensed under Creative Commons
%\documentclass[nocc]{template/openetcs_article}
\usepackage{lipsum,url}
\usepackage{supertabular}
\usepackage{multirow}
\usepackage{color, colortbl}
\definecolor{gray}{rgb}{0.8,0.8,0.8}
\usepackage[modulo]{lineno}
\graphicspath{{./template/}{.}{./images/}}

%New omponents for supporting #Glossary#
%\makeglossaries
%Glossary terms
%\loadglsentries{./glossary/openETCS-Latex-Glossary.tex}

\begin{document}
\frontmatter
\project{openETCS}

%Please do not change anything above this line
%============================



% The document metadata is defined below

%assign a report number here
\reportnum{OETCS/WP4/D0?}

%define your workpackage here
\wp{Work-Package 4: ``Governance and Validation''}

%set a title here
\title{How To Use the SRS Findings Reporting Template}

%set a subtitle here
\subtitle{Defined in Issue: \url{https://github.com/openETCS/validation/issues/52}}

%set the date of the report here
\date{November 2013}

%define a list of authors and their affiliation here

\author{Hekele Bernd}

\affiliation{Deutsche Bahn AG / DB Netz AG\\
  openETCS Project Group \\
  Voelckerstrasse 5\\
  80939 Muenchen, Germany \\
  \\
  eMail: bernd.hekele@deutschebahn.com \\
  WebSite: www.openETCS.org}


% define the coverart
\coverart[width=350pt]{openETCS_EUPL}

%define the type of report
\reporttype{Procedure}


\begin{abstract}
%define an abstract here
%  \lipsum[12-13]
This document introduces the process on how to document findings in the SRS and the template for documenting findings in  the SRS specification. We plan to use this information for reporting problems to the author of the specification. 
  
\end{abstract}

%=============================
\maketitle

%Modification history
%if you do not need a modification history table for your document simply comment out the eight lines below
%=============================
\section*{Modification History}
\tablefirsthead{
\hline 
\rowcolor{gray} 
Version & Section & Modification / Description & Author \\\hline}
\begin{supertabular}{| m{1.2cm} | m{1.2cm} | m{6.6cm} | m{4cm} |}
0.1 & All Parts & New Document & Bernd Hekele\\
 & & & \\\hline
\end{supertabular}


\tableofcontents
\listoffiguresandtables
\newpage
%=============================

%Uncomment the next line if you need line numbers for tracebility when the document is in review
%\linenumbers
%=============================


% The actual document starts below this line
%=============================
%The following subsections list the important glossary terms and the abbreviations used in %this document. 


\section{Introduction}
The openETCS ITEA project was explicitly introduced in order to look for ambiguities, incompletenesses and gaps in the framework  of the System Requirement Specifications. 

This sort of error shall be document in the project and reported to the specification entity being responsible for issuing the SRS.

In the openETCS project DB will take responsibility to collect the findings and report the findings via its board member to the author. 

Each finding has provide the information needed to be able to understand the finding and to react on the finding. Therefore, this template is giving a form to the specification of the findings.

\section{What is a finding}
A finding can be anything between a request for clarification and a system critical error in the specification. A finding can be resulting of SRS analysis, from implementation, from verification or from any other activity. Regardless of the source of the finding we need to use the same way to get the finding reported to the SRS owner.

\section{Reporting}
\begin{itemize}
\item For reporting of findings a special folder in the repository SSRS is used: \url{https://github.com/openETCS/SSRS/Findings}
\item In order to report a finding, you fill in the form, name it "SRS-Finding-<>identification number<>-<>short text<>
\item  Open a new issue with the title: "SRS Finding: <>short text<> where short text can be a SRS-Reference to the standard or similar.
\item You need to set the <>identification number<> for the finding. Use the number of the issue in SSRS you used for this finding.
\item Save the issue, update the document and push the document to the findings folder.
\item The issue will be used to keep track of the discussions following inside the openETCS community. 
\item For clarification, first, a railway expert from the openETCS will make an evaluation.
\item After clarification inside the openETCS community is completed, a decision is need on how this finding has to be handled. 
\item If there is no agreement between author and the expert escalation is to the PMB. The project-office will keep track on status and is responsible to make decisions transparent.
\end{itemize}

\section{Next Steps}
This document is a first draft. Discussion is needed within openETCS and with our interface to ERA.

It is planned to limit the size of this form to a minimum for a use within the project. The addidional information (mainly related to the ERA interface) will be added only in the cases when ERA is involved.

\newpage

\section{The Template}

\subsection{The Form}

\tablefirsthead{
\hline 
\rowcolor{gray} 
Item & Description of Finding \\\hline}
\begin{supertabular}{| m{4cm} |  m{10cm} |}
Identification number & \\\hline
State &  \\\hline
Headline & \\\hline
Impacted System & \\\hline
Reference Baseline Release & \\\hline
Documents and/or References  & \\\hline
Error/Enhancement & \\\hline
Problem/Need description & \\\hline
Supporting document(s) for problem/need description & \\\hline
Solution Proposal by submitter& \\\hline
Supporting document(s) for solution proposal & \\\hline
Agreed Solution & \\\hline
Supporting document(s) for agreed solution & \\\hline
Justification/Discussion for Solution & \\\hline
Supporting document(s) for Justification/Discussion for Solution & \\\hline
Preliminary Assessment of Benefits & \\\hline
Supporting document(s) for preliminary Assessment Benefits & \\\hline
Economic Evaluation & \\\hline
Supporting document(s) for Economic Evaluation & \\\hline
Submitting Recognised Organisation & \\\hline
Contact person name & \\\hline
Contact Person e-mail adress & \\\hline
Endorsed by Recognised Organisation(s) & \\\hline
Project Information & \\\hline
Submitter Reference Number  & \\\hline
List of assigned WG(s) & \\\hline
Severity & \\\hline
Target Baseline  & \\\hline
Reason for Error/Enhancement reclassification & \\\hline
Reason for rejection & \\\hline
Reason for postponement & \\\hline
Superseding CR & \\\hline
List of superseded CRs & \\\hline
Date of Submission & \\\hline
Last modification date & \\\hline
\end{supertabular}

\subsection{Definition of the Fields in the Form}
Most of the entries in the table can be added later in the process.

\begin{enumerate}

\item Identification number\\
refers to the issue number
\item State\\
can be one of: "New", "Accepted", "Rejected", "In clarification", "Presented"
\item Headline\\
a short title for the finding
\item Impacted System\\
can be either ETCS or a subfunction.
\item Reference Baseline Release\\
The Release of the SRS you are referring to (e.g., 3.3.0).
\item Documents and/or References \\
Identifies the location in the SRS as close as possible. If more documents are involved, please, use a list.
\item Error/Enhancement\\
\item Problem/Need description\\
Describe the problem as clear as possible.
\item Supporting document(s) for problem/need description\\
 If needed, the reference to a seperate document (e.g., testcase) can be added.
\item Solution Proposal by submitter\\
Short technical description of your proposal
\item Supporting document(s) for solution proposal\\
If needed, the reference to a seperate document (e.g., testcase) can be added.
\item Agreed Solution\\
This is documenting the change in the SRS
\item Supporting document(s) for agreed solution
Reference to additional document if needed.
\item Justification/Discussion for Solution\\
Will be filled in the process of discussion with the author.
\item Supporting document(s) for Justification/Discussion for Solution\\
Reference to additional document if needed.
\item Preliminary Assessment of Benefits\\
Will be filled in the process of discussion with the author.
\item Supporting document(s) for preliminary Assessment Benefits
\item Economic Evaluation\\
Will be filled in the process of discussion with the author.
\item Supporting document(s) for Economic Evaluation\\
will be added by projectoffice when relevant.
\item Submitting Recognised Organisation\\
will be added by projectoffice when relevant.
\item Contact person name\\
will be added by projectoffice when relevant.
\item Contact Person e-mail adress\\
will be added by projectoffice when relevant.
\item Endorsed by Recognised Organisation(s)\\
will be added by projectoffice when relevant.
\item Project Information\\
will be added by projectoffice when relevant.
\item Submitter Reference Number\\
\item List of assigned WG(s)\\
\item Severity\\
will be added by projectoffice when relevant.
\item Target Baseline \\
Will be filled in the process of discussion with the author.
\item Reason for Error/Enhancement reclassification\\
Will be filled in the process of discussion with the author.\\
\item Reason for rejection\\
Will be filled in the process of discussion with the author.
\item Reason for postponement\\
Will be filled in the process of discussion with the author.
\item Superseding CR\\
Will be filled in the process of discussion with the author.
\item List of superseded CRs\\
Will be filled in the process of discussion with the author.
\item Date of Submission\\
Time and Datestamp.
\item Last modification date\\
Time and Datestamp.

\end{enumerate}

%===================================================
%Do NOT change anything below this line

%\subsubsection{Glossary}

%\glossarystyle{long}
%\printglossary[title=]


%\subsubsection{Abbreviations}
%\printglossary[type=\acronymtype,title=]

\section{Glossary}

\end{document}
